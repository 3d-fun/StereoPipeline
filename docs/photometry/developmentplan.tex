\chapter{Development Plan}
\label{ch:developmentplan}

\begin{wrapfigure}{r}{0.4\textwidth}
  \vspace{-120pt}
  \begin{center}
    \includegraphics[width=0.4\textwidth]{graph/phoalgorithm}
  \end{center}
  \vspace{-20pt}
  \caption{Algorithm Outline.}
  \vspace{-10pt}
\end{wrapfigure}

\section{Algorithm Outline}

\begin{enumerate}
\item Stereo Processing
\item Initialize Photometry
  \begin{itemize}
    \item \textbf{Initialize DEMs and Blend} \hfill \\
      This should be done with the already provided The DEM variance found is optional written here.
      \texttt{image2plate} and \texttt{platereduce}.
    \item \textbf{Ingest DRGs} \hfill \\
      This involves thresholding the DRG for shadows and then solves
      for a grassfire weighting. The weighting and the masking are the
      same thing and should be stored inside the alpha channel for
      quick access. The masks used is optional written.
    \item \textbf{Seed Reflectance Images} \hfill \\
      Load up camera models, and use SPICE to determine Sun's
      location.
    \item \textbf{Seed Exposure Time} \hfill \\
      Somehow uses ratio of reflectance returns from previous step.
    \item \textbf{Seed Albedo Image} \hfill \\
      Use this equation:
      \[
      A_{ij}=\sum_{k}\frac{I^{k}_{ij}}{R^{k}_{ij}T^{k}}
      \]
  \end{itemize}
\item Iterate Solution
  \begin{itemize}
    \item \textbf{Re-estimate Exposure} \hfill \\
      \[
      \hat{T^{k}} = T^{k}+\frac{\sum_{ij}(I^{k}_{ij}-T^{k}A_{ij}R^{k}_{ij})A_{ij}R^{k}_{ij}S^{k}_{ij}}{\sum_{ij}(A_{ij}R^{k}_{ij}S^k_{ij})^{2}}
      \]
    \item \textbf{Re-estimate Albedo} \hfill \\
      \[
      \hat{A_{ij}} = A_{ij}+\frac{\sum_{k}(I^k_{ij}-T^kA_{ij}R^k_{ij})T^kR^k_{ij}S^{k}_{ij}}{\sum_{k}(T^kR^k_{ij}S^k_{ij})^2}
      \]
    \item \textbf{Re-estimate DEM} \hfill \\
      \emph{MAGIC!}
    \item \textbf{Re-calculate Reflectance} \hfill \\
      Refer back to the cameras.
    \item \textbf{Calculate Error} \hfill \\
      Error is used to determine if we're improving.
      \[
      \epsilon=\sum_{k}\sum_{ij}((I^k_{ij}-A_{ij}T^kR^k_{ij})S^k_{ij})^2
      \]
  \end{itemize}
\end{enumerate}

\section{Content of Files}

These files represent the results and working files from one session
of the Photometry Toolkit. One project is inside a
\emph{PhoBundle}. Here's the contents:

\begin{itemize}
  \item \textbf{Albedo.plate} \hfill \\
    PixelGrayA<uint8>
  \item \textbf{DEM.plate} \hfill \\
    PixelGrayA<int16>
  \item \textbf{DEMVariance.plate} \hfill \\
    PixelGrayA<int16>
  \item \textbf{Mask.plate} \hfill \\
    PixelGrayA<float32>
  \item \textbf{DRG.plate} \hfill \\
    PixelGrayA<uint8>
  \item \textbf{Reflectance.plate} \hfill \\
    PixelGrayA<float32>
  \item \textbf{Error.plate} \hfill \\
    PixelGrayA<uint8>
\end{itemize}

The platefiles Albedo, DEM, DEMVariance, and Mask are very similar and
operate like normal plates do. These files are global image mosaics
where the highest transaction ID represents the current working
version. DEMVariance and Mask are both helpful plate files that are
used for reporting; but they're not used again after the
initialization step.

The platefiles DRG, Reflectance, and Error are a little more
tricky. They represent multiple K Camera's and their history. Their
transaction IDs involve a little math so that $ID_{transaction} =
100*k+i$. Where $k$ is the camera number and $i$ is the iteration
number. We provide only enough room for a hundred transactions.

There's also one other file, the \emph{PhoFile} which contains most of
the project information. If your familiar with the original Photometry
module, this is the model params and the global params stitched
together. I imagine this this file will probably use
proto-buffers. Here's an outline of the contents.

\begin{itemize}
  \item PhoFile \hfill \\
    \begin{description}
      \item[Project Message] \hfill \\
        This contains Datum along with various information transaction
        IDs.
      \item[Repeated Camera Message] \hfill \\
        This contains the exposure time along with the Vector3 for the
        Sun Position and the Spacecraft Position.
    \end{description}
\end{itemize}

\section{Executables}

\begin{wrapfigure}{r}{0.5\textwidth}
  \vspace{-50pt}
  \begin{center}
    \includegraphics[width=0.5\textwidth]{graph/executables}
  \end{center}
  \caption{Executables}
\end{wrapfigure}

Everything inside the Photometry Toolkit will hopefully fit
comfortably inside 2 python scripts, \texttt{phoinit.py} and
\texttt{phosolve.py}. As expected, \texttt{phoinit.py} creates all the
files to start with. \texttt{phosolve.py} performs all the
iterations. In the end when the user wants the results, they must
perform \texttt{snapshot} or \texttt{plate2tile} to get the results.

\subsection{Individual calls}

Eventually the the python scripts should negate the need for
this. Here are the actual executables called and the settings to use.

\begin{verbatim}
  > sudo $(START UP AMQP COMMAND)
  > (in a different terminal) index_server $(PROJECT DIR)
  > cd $(DEM_DIR)
  > ls *DEM.tif | xargs -n 1 -P 10 grassfirealpha --nodata -10000
  > ls *grass.tif | xargs -n 1 -P 10 image2plate -o
      pf://index/DEM_blend.plate --file tif -m equi
  > for i in {0..$(NUM LEVELS)}; do platereduce --end_t 1999 -t 2000
      pf://index/DEM_blend.plate -l $i & done
  > plate2plate -o pf://index/DEM.plate -i pf://index/DEM_blend.plate
      --filter identity --bottom 10 --skim
  > (in other terminal, kill index_server)
  > rm -rf DEM.plate
  > (in other terminal, restart index_server)
  > j=0; for i in $(ls *DRG.tif);
      do phodrg2plate $i $((j*10)) & j=$((j+1)); done
  >
\end{verbatim}
