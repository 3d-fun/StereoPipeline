\chapter{Development Plan}
\label{ch:developmentplan}

\begin{wrapfigure}{r}{0.3\textwidth}
  \vspace{-40pt}
  \begin{center}
    \includegraphics[width=0.25\textwidth]{graph/phoalgorithm}
  \end{center}
  \caption{Toolkit Outline}
  \vspace{-90pt}
\end{wrapfigure}

This provides a general outline of the Photometry Toolkit and how I
currently envision it working.

\section{Algorithm Outline}

\begin{enumerate}
\item Stereo Processing
\item Initialize Photometry
  \begin{itemize}
    \item \textbf{Initialize DEMs and Blend} \hfill \\
      This should be done with the already provided The DEM variance found is optional written here.
      \texttt{image2plate} and \texttt{platereduce}.
    \item \textbf{Ingest DRGs} \hfill \\
      This involves thresholding the DRG for shadows and then solves
      for a grassfire weighting. The weighting and the masking are the
      same thing and should be stored inside the alpha channel for
      quick access. The masks used is optional written.
    \item \textbf{Seed Reflectance Images} \hfill \\
      Load up camera models, and use SPICE to determine Sun's
      location.
    \item \textbf{Seed Exposure Time} \hfill \\
      Somehow uses ratio of reflectance returns from previous step.
    \item \textbf{Seed Albedo Image} \hfill \\
      Use this equation:
      \[
      A_{ij}=\sum_{k}\frac{I^{k}_{ij}}{R^{k}_{ij}T^{k}}
      \]
  \end{itemize}
\item Iterate Solution
  \begin{itemize}
    \item \textbf{Re-estimate Exposure} \hfill \\
      \[
      \hat{T^{k}} = T^{k}+\frac{\sum_{ij}(I^{k}_{ij}-T^{k}A_{ij}R^{k}_{ij})A_{ij}R^{k}_{ij}S^{k}_{ij}}{\sum_{ij}(A_{ij}R^{k}_{ij}S^k_{ij})^{2}}
      \]
    \item \textbf{Re-estimate Albedo} \hfill \\
      \[
      \hat{A_{ij}} = A_{ij}+\frac{\sum_{k}(I^k_{ij}-T^kA_{ij}R^k_{ij})T^kR^k_{ij}S^{k}_{ij}}{\sum_{k}(T^kR^k_{ij}S^k_{ij})^2}
      \]
    \item \textbf{Re-estimate DEM} \hfill \\
      \emph{MAGIC!}
    \item \textbf{Re-calculate Reflectance} \hfill \\
      Refer back to the cameras.
    \item \textbf{Calculate Error} \hfill \\
      Error is used to determine if we're improving.
      \[
      \epsilon=\sum_{k}\sum_{ij}((I^k_{ij}-A_{ij}T^kR^k_{ij})S^k_{ij})^2
      \]
  \end{itemize}
\end{enumerate}

\section{Content of Files}

These files represent the results and working files from one session
of the Photometry Toolkit. One project is inside a directory
masquerading as file with extension \texttt{.ptk}. Here is a list of
the possible contents contents. These files are only created one need.

\begin{itemize}
  \item \textbf{photometrytk.dat} \hfill \\
    This is a binary file written with google protobuf. This contains
    all the meta information about a PhoTK project. It is also what is
    served and modified as a service. The reason for this extra
    complexity is that multiple processes on multiple machines will be
    trying to access this file.
  \item \textbf{photometrytk.bak} \hfill \\
    Back up file of \texttt{photometrytk.dat} that is written
    periodically in the case of a fault. Users can rename this file to
    \texttt{photometrytk.dat} if they think the original file is
    corrupt.
  \item \textbf{Albedo.plate} \hfill \\
    Is a PixelGrayA<uint8> platefile. This contains the final result of
    a mosaic.
  \item \textbf{DEM.plate} \hfill \\1
    Is a PixelGrayA<int16> platefile. This contains the input DEM
    platefile. \emph{There is currently not a complete idea about how this
    is to be made or used by PhoTK.}
  \item \textbf{DEMVariance.plate} \hfill \\
    Is a PixelGrayA<int16> platefile. \emph{This is still in design.}
  \item \textbf{DRG.plate} \hfill \\
    Is a PixelGrayA<uint8> platefile. It contains the input images.
  \item \textbf{Reflectance.plate} \hfill \\
    Is a PixelGrayA<uint8> platefile. \emph{This is still in design.}
  \item \textbf{Error.plate} \hfill \\
    Is a PixelGrayA<uint8> platefile. \emph{This is still in design.}
\end{itemize}

The platefiles Albedo, DEM, and DEMVariance are very similar and
operate like normal plates do. These files are global image mosaics
where the highest transaction ID represents the current working
version.

%% DEMVariance and Mask are both helpful plate files that are
%% used for reporting; but they're not used again after the
%% initialization step.

The platefiles DRG, Reflectance, and Error are a little more
tricky. They represent multiple K Cameras and their history. Their
transaction IDs involve a little math so that $ID_{transaction} =
100*k+i$. Where $k$ is the camera number and $i$ is the iteration
number. We provide only enough room for a hundred transactions.

%% There's also one other file, the \emph{PhoFile} which contains most of
%% the project information. If your familiar with the original Photometry
%% module, this is the model params and the global params stitched
%% together. I imagine this this file will probably use
%% proto-buffers. Here's an outline of the contents.

%% \begin{itemize}
%%   \item PhoFile \hfill \\
%%     \begin{description}
%%       \item[Project Message] \hfill \\
%%         This contains Datum along with various information transaction
%%         IDs.
%%       \item[Repeated Camera Message] \hfill \\
%%         This contains the exposure time along with the Vector3 for the
%%         Sun Position and the Spacecraft Position.
%%     \end{description}
%% \end{itemize}

\section{Executables}

%% \begin{wrapfigure}{r}{0.5\textwidth}
%%    \begin{center}
%%      \includegraphics[width=0.5\textwidth]{graph/exec_init}
%%    \end{center}
%%    \caption{Initialization Executables}
%% \end{wrapfigure}

Everything inside the Photometry Toolkit will hopefully fit
inside 2 python scripts, \texttt{phoinit.py} and
\texttt{phosolve.py}. As expected, \texttt{phoinit.py} creates all the
files to start with. \texttt{phosolve.py} performs all the
iterations. In the end when the user wants the results, they must
perform \texttt{snapshot} or \texttt{plate2tile} to get the results.

At this time of writing, only \texttt{phosolve.py} has been
written. Everything is still largely manual and requires some thought
by the user. The rest of this section outlines how to do some tasks
that will hopefully add up to a full PhotometryTK session.

\subsection{Creating a mosaicked DEM}

This step is not currently used, but I thought it was worth noting as
there are a few tricks. Mosaicking images in platefiles,is currently
done with a utility called \texttt{platereduce}. It is good for
calculating weighted averages or weighted variances. Calculating
variances is mostly just useful for calculating DEM precision. If you
are master of \texttt{snapshot}, you'll find that using
\texttt{platereduce} to be very similar.

I assume you have a working \texttt{index\_server} setup. The first
steps are to apply \texttt{grassfirealpha} to all input images. This
creates an alpha channel with feathered weights. The weights are
highest value in the center of the image and lowest value near the
edge. After that we'll enter all the alpha'd images into a single
platefile. I'm using \texttt{xargs} to make this step multiprocess so
I can utilize my entire machine. The \texttt{-P} option is the number
of simultaneous processes to have running. Scale that setting to the
appropriate value for your machine.

\begin{verbatim}
  > echo *DEM.tif | xargs -n1 -P15 grassfirealpha
  > echo *grass.tif | xargs -n1 -P15 image2plate -o pf:///index/DEM_blend.plate \
    --file tif -m equi
\end{verbatim}

Notice on \texttt{image2plate} that I set the output platefile to a
TIF. This is important as the default option of PNG will shave bits on
alpha'd values. I'm also using the 'equi' output option for an
equirectangular projection. The default is to use a TOAST projection.

We are now ready to run \texttt{platereduce}. The options
\texttt{begin\_transaction} and \texttt{end\_transaction} specify the
range of transaction IDs to read as input. The \texttt{transaction-id}
option specify the output transaction ID to use.

Reducing plate can be a very slow task, because of this we'll be doing
it in parallel with multiple commands. This requires us to tell the
\texttt{index\_server} by hand when we are starting a transaction and
ending one. Notice how this is done if the following example.

\begin{verbatim}
  > platereduce pf:///index/DEM_blend.plate -t 2000 --start "Averaging DEMs"
  > for i in {0 .. $(NUM_LEVELS)}; do platereduce --end_t 1999 -t 2000
       pf:///index/DEM_blend.plate -f WeightedAvg & done
  > platereduce pf:///index/DEM_blend.plate -t 2000 --finish
\end{verbatim}

At this point the averaged DEM should be in transaction ID 2000 and it
averaged all transaction IDs between 0 and 1999.

\subsection{Creating a PhotometryTK project}

Creating a project is done with the command \texttt{phoinitfile}. With
this command you set what reflectance type you are solving for. You
also specify which map projection to use for the project file. Current
options are TOAST, Plate Equirectangular, and Polar. You also specify
the number iterations to use for solving for the solution
here. \emph{The number of iterations is not changeable after this
  step.}

\begin{verbatim}
  > phoinitfile usgspolar -m polar --max 100
\end{verbatim}

At this point we are ready to start serving the project out. This is
accomplished with the utility called \texttt{ptk\_server}. The PTK
server is both the platefile index server and PhoTK project file
server. \emph{You will not be able to run another instance reliably of
  \texttt{index\_server} at the same time as \texttt{ptk\_server}.} It
is possible that you might have an exchange name
collision. \emph{Currently, PhoTK does not support the use of ZMQ
  urls.}

\begin{verbatim}
  > ptk_server usgspolar.ptk --url pf:///polar
\end{verbatim}

The PTK server only serves one project at a time, therefor the first
argument is the input project file \emph{(that is secretly a
  directory)}. The second argument specify the URL prefix we will be
using for this project. A later points we will be needing our
project's URL. It is the combination of the URL prefix and then the
project file's name. For this example it is: \texttt{pf:///polar/usgspolar.ptk}

At this point we are ready to start inserting our input images into
the project. This is done with \texttt{phodrg2plate}. This is a clone
of \texttt{image2plate} but it gets most of its settings from the
served project file. Here's an example of me inserting files.

\begin{verbatim}
  > echo *.tif | xargs -n1 -P15 phodrg2plate pf:///polar/usgspolar.ptk
\end{verbatim}

If you look inside the project file, you'll see that the platefile
\texttt{DRG.plate} has been created.

\subsection{Solving a PhotometryTK project}

%%\begin{wrapfigure}{r}{0.4\textwidth}
  \begin{center}
    \includegraphics[width=0.4\textwidth]{graph/exec_solve}
  \end{center}
%%\caption{Iteration Executables}
%%\end{wrapfigure}

Running the actually iterations of solving is pretty easy since it is
currently hidden away in a python script. The file script
\texttt{phosolve.py} does all the steps of re-estimating exposure and
updating albedo. In the future, calculating reflectance and
shape-from-shading will be inserted as optional steps in this file.

\begin{verbatim}
  > phosolve.py --threads 15 --level 9 --iter 100 pf:///polar/usgspolar.ptk
\end{verbatim}

The level option is just what level of the image pyramid we wish to
perform our calculations. The lower the number \emph{(the higher up
  the pyramid)} will increase the speed at which things are
processed. However there is more likely a chance of error. I usually
run my sessions at about two thirds down the pyramid.

After this step is completed you will find that \texttt{Albedo.plate}
has been created in your project file. Viewing it can be
tricky. \emph{(In my example, the Albedo's url will be
  pf:///polar\_index/Albedo.plate.)} However you will find that only
one level of the pyramid has actually be populated. That makes it
difficult to locate that solution. Things are easier for the viewer if
you mipmap the solution up the pyramid.

\begin{verbatim}
  > pmipmap --level 9 --threads 15 pf:///polar_index/Albedo.plate
\end{verbatim}

Viewing should now be much easier in VWV.

\begin{verbatim}
  > vwv pf:///polar_index/Albedo.plate
\end{verbatim}

\subsection{Running once at high resolution and extracting result}

Meh.

%% \subsection{An example run of PhotometryTK}

%% Eventually the the python scripts should negate the need for
%% this. Here are the actual executables called and the settings to
%% use. Delete this section some time in the future.

%% \begin{verbatim}
%%   > sudo $(START UP AMQP COMMAND)
%%   > (in a different terminal) index_server $(PROJECT DIR)
%%   > cd $(DEM_DIR)
%%   > ls *DEM.tif | xargs -n 1 -P 10 grassfirealpha --nodata -10000
%%   > ls *grass.tif | xargs -n 1 -P 10 image2plate -o
%%       pf://index/DEM_blend.plate --file tif -m equi
%%   > for i in {0..$(NUM LEVELS)}; do platereduce --end_t 1999 -t 2000
%%       pf://index/DEM_blend.plate -l $i & done
%%   > plate2plate -o pf://index/DEM.plate -i pf://index/DEM_blend.plate
%%       --filter identity --bottom 10 --skim
%%   > (in other terminal, kill index_server)
%%   > rm -rf DEM_blend.plate
%%   > (in other terminal, restart index_server)
%%   > phoinitfile debug_apollo
%%   > // Below can be ran in parallel
%%   > echo *.tif | xargs -n1 echo |
%%       xargs -n1 -P10 phodrg2plate pf://ptk/project.ptk
%%   > echo {0..9} | xargs -n1 echo |
%%       xargs -n1 -P10 -I{} phoitalbedo -j {} -n 10 pf://ptk/project.ptk
%%   > mipmap pf://index/Albedo.plate
%% \end{verbatim}
