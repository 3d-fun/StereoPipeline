\chapter{Modifying SURF to output VW match files}

What follows is the output of \texttt{diff} between the original SURF
v1.0.9 code and the modified code to write a Vision Workbench style
match file. This Appendix only applies to Linux and potentially
Windows users. SURF is currently unavailable for Mac OSX.

\section{Patch to match.cpp}

\begin{verbatim}
--- orginal_code/SURF-V1.0.9/match.cpp	2006-12-20 03:28:30.000000000 -0600
+++ match.cpp	2009-03-11 13:26:07.000000000 -0500
@@ -31,7 +31,7 @@
  */
 
 #include <vector>
-#include <string>
+#include <string.h>
 #include <iostream>
 #include <fstream>
 #include <cmath>
@@ -39,12 +39,10 @@
 #include "ipoint.h"
 
 // Define to compile with PGM output
-#define GRAPHICS
+//#define GRAPHICS
 
-#ifdef GRAPHICS
 #include "image.h"
 #include "imload.h"
-#endif
 
 using namespace std;
 using namespace surf;
@@ -69,6 +67,7 @@
 	int match = -1;
 
 	for (unsigned i = 0; i < ipts.size(); i++) {
+
 		// Take advantage of Laplacian to speed up matching
 		if (ipts[i].laplace != ip1.laplace)
 			continue;
@@ -90,6 +89,30 @@
 	return -1;
 }
 
+// Writes the interestpoint match for later reading
+inline void write_ip_record(std::ofstream &f, Ipoint const& p){
+  float buffer_f = (float)p.x;
+  f.write((char*)&(buffer_f), sizeof(float));       //x
+  buffer_f = (float)p.y;
+  f.write((char*)&(buffer_f), sizeof(float));       //y
+  int buffer_i = (int)p.x;
+  f.write((char*)&(buffer_i),   sizeof(int));         //ix
+  buffer_i = (int)p.y;
+  f.write((char*)&(buffer_i),   sizeof(int));         //iy
+  buffer_f = (float)p.ori;
+  f.write((char*)&(buffer_f), sizeof(float));     //ori
+  buffer_f = (float)p.scale;
+  f.write((char*)&(buffer_f), sizeof(float));     //scale
+  buffer_f = (float)p.strength;
+  f.write((char*)&(buffer_f),sizeof(float)); //interest
+  bool buffer_b = false;
+  f.write((char*)&(buffer_b),sizeof(bool));  //polarity
+  f.write((char*)&(buffer_i),sizeof(unsigned)); // octave
+  f.write((char*)&(buffer_i),sizeof(unsigned)); // scale_lvl
+  buffer_i = 0;
+  f.write((char*)&(buffer_i),     sizeof(int));         //size of descriptor... nothing...                                                                                                
+}  
+
 // Find all possible matches between two images
 vector< int > findMatches(const vector< Ipoint >& ipts1, const vector< Ipoint >& ipts2) {
 	vector< int > matches(ipts1.size());
@@ -98,9 +121,7 @@
 		int match = findMatch(ipts1[i], ipts2);
 		matches[i] = match;
 		if (match != -1) {
-			cout << " Matched feature " << i << " in image 1 with feature "
-				<< match << " in image 2." << endl;
-			c++;
+		  c++;
 		}
 	}
 	cout << " --> Matched " << c << " features of " << ipts1.size() << " in image 1." << endl;
@@ -127,10 +148,10 @@
 	// Load the interest points in Mikolajczyk's format
 	for (unsigned n = 0; n < count; n++) {
 		// circular regions with diameter 5 x scale
-		float x, y, a, b, c;
+	  float x, y, a, b, c, ori;
 
 		// Read in region data, though not needed for actual matching
-		ipfile >> x >> y >> a >> b >> c;
+	  ipfile >> x >> y >> a >> b >> c >> ori;
 
 		float det = sqrt((a-c)*(a-c) + 4.0*b*b);
 		float e1 = 0.5*(a+c + det);
@@ -142,6 +163,7 @@
 		ipts[n].x = x;
 		ipts[n].y = y;
 		ipts[n].scale = sc/2.5;
+		ipts[n].ori = ori;
 
 		// Read in Laplacian
 		ipfile >> ipts[n].laplace;
@@ -159,7 +181,7 @@
             (y1 >= im->getHeight() && y2 >= im->getHeight()))
 		return;
 
-	bool steep = std::abs(y2 - y1) > std::abs(x2 - x1);
+	bool steep = abs(y2 - y1) > abs(x2 - x1);
 	if (steep) {
 		int t;
 		t = x1;
@@ -182,7 +204,7 @@
 	}
 
 	int deltax = x2 - x1;
-	int deltay = std::abs(y2 - y1);
+	int deltay = abs(y2 - y1);
 
 	int error = 0;
 	int y = y1;
@@ -215,12 +237,14 @@
 
 int main(int argc, char **argv) {
 	Image *im1, *im2;
-#ifdef GRAPHICS
+
 	ImLoad ImageLoader;
 	vector< Ipoint > ipts1, ipts2;
 	bool drawc = false;
-#endif
+
 	char ofname[100];
+	string matchname;
+	matchname.clear();
 
 	im1 = im2 = NULL;
 	ofname[0] = 0;
@@ -232,7 +256,6 @@
 			loadIpoints(argv[++arg], ipts1);
 		if (! strcmp(argv[arg], "-k2"))
 			loadIpoints(argv[++arg], ipts2);
-#ifdef GRAPHICS
 		if (! strcmp(argv[arg], "-im1"))
 			im1 = ImageLoader.readImage(argv[++arg]); 
 		if (! strcmp(argv[arg], "-im2"))
@@ -241,12 +264,13 @@
 			strcpy(ofname, argv[++arg]);
 		if (! strcmp(argv[arg], "-c"))
 			drawc = true;
-#endif
+		if (! strcmp(argv[arg], "-m"))
+		  matchname = argv[++arg];
 	}
 
 	if (ipts1.size() == 0 || ipts2.size() == 0) {
 		cout << "Usage:" << endl;
-		cout << " match -k1 out1.surf -k2 out2.surf -im1 img1.pgm -im2 img2.pgm -o out.pgm" << endl << endl;
+		cout << " match -k1 out1.surf -k2 out2.surf -im1 img1.pgm -im2 img2.pgm -o out.pgm -m output.match" << endl << endl;
 		cout << "For each feature in first descriptor file, find best in second according to "
 			<< "nearest neighbor ratio strategy. Display matches in out.pgm, generated "
 			<< "from img1.pgm and img2.pgm. Use -c to draw crosses at interest points." << endl;
@@ -255,7 +279,33 @@
 
 	vector< int > matches = findMatches(ipts1, ipts2);
 
-#ifdef GRAPHICS
+	// Determining if to save a match file
+        if (matchname.size()) {
+          int count = 0;
+          for (unsigned i = 0; i < matches.size(); ++i){
+            if (matches[i] != -1)
+              ++count;
+          }
+          if (count > 0 ) {
+            std::ofstream outputFile(matchname.c_str(), std::ios::out);
+            outputFile.write((char*)&count, sizeof(int));
+            outputFile.write((char*)&count, sizeof(int));
+            //Writing left side..
+            for (unsigned i = 0; i < matches.size(); ++i){ 
+              if (matches[i] != -1) {
+                write_ip_record(outputFile,ipts1[i]);
+              } 
+            }
+            //Writing right side..
+            for (unsigned i = 0; i < matches.size(); ++i){
+              if (matches[i] != -1) {
+                write_ip_record(outputFile,ipts2[matches[i]]);
+              }  
+            } 
+          } 
+        } 
+
+
 	if (im1 != NULL && im2 != NULL && ofname[0] != 0) {
 		Image res(max(im1->getWidth(), im2->getWidth()), im1->getHeight() + im2->getHeight());
 		for (int x = 0; x < im1->getWidth(); x++)
@@ -283,7 +333,6 @@
 
 		ImageLoader.saveImage(ofname, &res);
 	}
-#endif
 	
 	return 0;
 }

\end{verbatim}

\section{How to apply and compile}

First move to the directory containing your copy of the SURF v1.0.9
code. Then copy the contents of the section above into a newly created
file called, \verb=match_cpp.diff=. Alternatively a copy of
\verb=match_cpp.diff= is included in the Appendix directory that is
inside the directory containing this documentation. At this point you
are ready to start running the following commands.

\begin{verbatim}
        patch < match_cpp.diff
        make match.ln
\end{verbatim}

\definecolor{lgray}{gray}{0.95}
\begin{center}
\fcolorbox{black}{lgray}{ \begin{minipage}{5.5in} 

    \emph{Note:} \\ If you are unfortunate enough to run into an error
    such as \textit{g++-4.0.2: Command not found}, don't worry. Edit
    \texttt{Makefile} at line 10 and 11 to refer to \texttt{g++}
    instead of \texttt{g++-4.0.2}. \\ \\ Also since you've incurred
    that error, you'll probably need to add an include to
    \texttt{<stdlib.h>} in \texttt{imload.cpp} in the same
    directory. This all stems from differences in using a newer
    version of g++.
\end{minipage}}
\end{center}

\section{Example of using SURF}

For this example it is assumed you have a directory containing two
images named \verb=m1000254.png= and \verb=r0901059.png= like in the
example found in Section 5.3.1.

SURF code only works with images in the grayscale format PGM. A free
Linux utility to convert the images is \texttt{mogrify}. That utility
is a part of the package \texttt{imagemagick} and is likely to be
available in most package managers.

Below is the commands to take an input of PNG files, process them with
SURF, and then finally create a match file which can be used by the
likes of \texttt{isis\_adjust}.

\begin{verbatim}
        mogrify -format pgm m1000254.png r0901059.png
        surf.ln -i m1000254.pgm -o m1000254.surf
        surf.ln -i r0901059.pgm -o r0901059.surf
        match.ln -k1 m1000254.surf -k2 r0901059.surf
                 -im1 m1000254.pgm -im2 r0901059.pgm
                 -o out.pgm -m m1000254__r0901059.match
        rm m1000254.pgm r0901059.pgm *.surf
\end{verbatim}

\begin{center}
\fcolorbox{black}{lgray}{ \begin{minipage}{5.5in} 

    It is important to note that though SURF is very good at
    performing matches it does not perform a step of RANSAC with it's
    output. There may be a couple of outliers.
\end{minipage}}
\end{center}


\chapter{Python Batch Processing}

Below is a Python script to process a large number of Apollo Metric
Camera pairs. This is meant only to serve as an example and it can be
modifed to run other commands.

\begin{verbatim}
#!/usr/bin/python

import glob;
import os;
import subprocess;

num_cores = 4;
joblist = [];
output_dir = "stereo";

def add_job( job ):
    if (len(joblist) >= num_cores):
        joblist[0].wait();
        joblist.pop(0);
    print job;
    joblist.append(subprocess.Popen(job,shell=True))
    return;

def run_cmd( left_file, right_file ):
    h_l = left_file.split(".");
    h_r = right_file.split(".");

    cmd = "stereo "+left_file+" "+right_file+" "+h_l[0]+".isis_adjust "+h_r[0]+".isis_adjust "+output_dir+"/"+h_l[0]+"__"+h_r[0]+" --threads 1"
    add_job(cmd);


files = glob.glob("*.cub");
files = sorted(files);
for left in range(0, len(files)):

    #determing how forward to look
    range_right = left + 2;
    if ( range_right > len(files) ):
        range_right = len(files);

    for right in range(left+1, range_right):
        run_cmd( files[left], files[right] );

for j in joblist:
    j.wait();

print "Done processing stereo pairs"

\end{verbatim}
