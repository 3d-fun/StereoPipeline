\chapter{Pinhole Models}
\label{chapter:pinholemodels}


Ames Stereo Pipeline supports a generic pinhole camera model with
several generic lens distortion models which cover common 
calibration methods.  The generic pinhole model uses the following
parameters:   

\begin{itemize}{}
\item  \textit{fx} = The focal length in horizontal units.
\item  \textit{fy} = The focal length in vertical units.
\item  \textit{cx} = The horizontal offset of the principal 
point of the camera in the image plane.
\item  \textit{cy} = The vertical offset of the principal 
point of the camera in the image plane.
\item  \textit{pitch} = The size of each pixel in the units used to specify
the four parameters listed above.  This will usually either be 1.0 if they
are specified in pixel units or alternately the size of a pixel in millimeters.
\end{itemize}

Along with the basic pinhole camera parameters, a lens distortion model
can be added.  Note that the units used in the distortion model must
match the units used for the parameters listed above.  For example, if the
camera calibration was performed using units of millimeters the focal lengths etc. must be given in
units of millimeters and the pitch must be equal to the size of each pixel in millimeters.
The following lens distortion models are currently supported:

\begin{itemize}{}
\item  \textbf{Null} = A placeholder model that applies no distortion.

\item  \textbf{Tsai} = A common distortion model similar to the one used by OpenCV and THEIA.
 This model uses the following parameters:
  \begin{description}
    \item \textit{K1, K2} = Radial distortion parameters.
    \item \textit{P1, P2} = Tangential distortion parameters.
  \end{description}
\hfill \\ The following equations describe the distortion:
\[ r^{2} = x^{2} + y^{2} \]
\[ x(distorted) = x*(K_{1}r^{2} + K_{2}r^{4} + 2P_{1}y + P_{2}(\frac{r^{2}}{x} + 2x)) \]
\[ y(distorted) = y*(K_{1}r^{2} + K_{2}r^{4} + 2P_{2}x + P_{1}(\frac{r^{2}}{y} + 2y))  \]
\hfill \\ References:
\begin{description}
   \item Roger Tsai, A Versatile Camera Calibration Technique for a High-Accuracy 3D
          Machine Vision Metrology Using Off-the-shelf TV Cameras and Lenses
\end{description}
\hfill \\ Note that this model uses normalized pixel units.

\item  \textbf{Adjustable Tsai} = A variant of the Tsai model where any number of \textit{K} terms and a skew term (alhpa) can be used.

\item  \textbf{Brown-Conrady} = An older model based on a centering angle.
  \begin{description}
    \item \textit{K1, K2, K3} = Radial distortion parameters.
    \item \textit{P1, P2} = Tangential distortion parameters.
    \item \textit{xp, yp} = Principal point offset.
    \item \textit{B1, B2} = Unused parameters.
  \end{description}
\hfill \\ The following equations describe the distortion:
\[ x = x(distorted) - xp \]
\[ y = y(distorted) - yp \]
\[ r^{2} = x^{2} + y^{2} \]
\[ dr = K_{1}r^{3} + K_{2}r^{5} + K_{3}r^{7} \]
\[ x(undistorted) = x + x\frac{dr}{r} + P_{1}(r^{2} +2x^{2}) + 2P_{2}xy \]
\[ y(undistorted) = y + y\frac{dr}{r} + P_{2}(r^{2} +2y^{2}) + 2P_{1}xy \]
\hfill \\ Note that this model uses non-normalized pixel units.

\hfill \\ References:
\begin{description}
   \item Decentering Distortion of Lenses - D.C. Brown, 
          Photometric Engineering, pages 444-462, Vol. 32, No. 3, 1966
   \item Close-Range Camera Calibration - D.C. Brown, 
          Photogrammetric Engineering, pages 855-866, Vol. 37, No. 8, 1971
\end{description}

\item  \textbf{Photometrix} = A model matching the conventions used by the Australis software from Photometrix.
 This model uses the following parameters:
  \begin{description}
    \item \textit{K1, K2, K3} = Radial distortion parameters.
    \item \textit{P1, P2} = Tangential distortion parameters.
    \item \textit{xp, yp} = Principal point offset.
    \item \textit{phi} = Tangential distortion angle in radians.
  \end{description}
\hfill \\ The following equations describe the distortion:
\[ x = x(distorted) - xp \]
\[ y = y(distorted) - yp \]
\[ r^{2} = x^{2} + y^{2} \]
\[ dr = K_{1}r^{3} + K_{2}r^{5} + K_{3}r^{7} \]
\[ x(undistorted) = x + x\frac{dr}{r} - (P_{1}r^{2} +P_{2}r^{4})sin(phi) +  \]
\[ y(undistorted) = y + y\frac{dr}{r} + (P_{1}r^{2} +P_{2}r^{4})cos(phi) \]
\hfill \\ Note that this model uses non-normalized pixel units.

\end{itemize}

\hrule
\section{File Format}
\label{file_format}
\bigskip

ASP pinhole model files are written in an easy to work with plain text format 
using the extension \texttt{.tsai}.
A sample file is shown below:

\begin{verbatim}
VERSION_3
fu = 28.429
fv = 28.429
cu = 17.9712
cv = 11.9808
u_direction = 1  0  0
v_direction = 0  1  0
w_direction = 0  0  1
C = 266.943 -105.583 -2.14189
R = 0.0825447 0.996303 -0.0238243 -0.996008 0.0832884 0.0321213 0.0339869 0.0210777 0.9992
pitch = 0.0064
Photometrix
xp = 0.004
yp = -0.191
k1 = 1.31024e-04
k2 = -2.05354e-07
k3 = -5.28558e-011
p1 = 7.2359e-006
p2 = 2.2656e-006
b1 = 0.0
b2 = 0.0
\end{verbatim}

The first half of the file is the same for all pinhole models:

\begin{itemize}{}
\item  \texttt{VERSION\_X} = A header line used to track the format of the file.
\item  \texttt{fu, fv, cu, cv} = The first four intrinsic parameters described in the previous section.
\item  \texttt{u, v, and w\_direction} = These lines allow an additional permutation of the 
axes of the camera coordinates.  By default, the positive column direction aligns with x, the
positive row direction aligns with y, and downward into the image aligns with z.
\item  \texttt{C} = The location of the camera center, usually in the geocentric coordinate system (GCC).
\item  \texttt{R} = The rotation matrix describing the camera's pose in the coordinate system.
\item  \texttt{pitch} = The pitch intrinsic parameter described in the previous section.
\end{itemize}

The second half of the file describes the lens distortion model being used.  The name of the
distortion model appears first, followed by a list of the parameters for that model.  The number
of parameters may be different for each distortion type.  Samples of each format are shown below:

\begin{itemize}{}
\item  \textbf{Null}
\begin{verbatim}
NULL
\end{verbatim}

\item  \textbf{Tsai}
\begin{verbatim}
TSAI
k1 = 1.31024e-04
k2 = -2.05354e-07
p1 = 0.5
p2 = 0.4
\end{verbatim}

\item  \textbf{Adjustable Tsai}
\begin{verbatim}
AdjustableTSAI
Radial Coeff: Vector3(1.31024e-04, 1.31024e-07, 1.31024e-08)
Tangential Coeff: Vector2(-2.05354e-07, 1.05354e-07)
Alpha = 0.4
\end{verbatim}

\item  \textbf{Brown-Conrady}
\begin{verbatim}
BrownConrady
xp = 0.5
yp = 0.4
k1 = 1.31024e-04
k2 = -2.05354e-07
k3 = 1.31024e-08
p1 = 0.5
p2 = 0.4
phi = 0.001
\end{verbatim}

\item  \textbf{Photometrix}
\begin{verbatim}
Photometrix
xp = 0.004
yp = -0.191
k1 = 1.31024e-04
k2 = -2.05354e-07
k3 = -5.28558e-011
p1 = 7.2359e-006
p2 = 2.2656e-006
b1 = 0.0
b2 = 0.0
\end{verbatim}

\end{itemize}{}

For several years Ames Stereo Pipeline generated pinhole files in the binary \texttt{.pinhole} format.
That format has been deprecated and can be read but will no longer be written.  
At some point in the future support for that file format will be dropped.

Also in the past Ames Stereo Pipeline has generated a shorter version of the current file
format, also with the extension \texttt{.tsai}, which only supported the TSAI lens distortion model.
Existing files in that format can still be used by ASP.









