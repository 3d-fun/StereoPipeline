\chapter{The {\tt stereo.default} File}
\label{ch:stereodefault}

The \texttt{stereo.default} file contains configuration parameters
that the \texttt{stereo} program uses to process images.  The
\texttt{stereo.default} file is loaded from the current working
directory when you run \texttt{stereo} unless you specify a different
file using the \texttt{-s} option.  Run \texttt{stereo --help} for
more information.

Below we will walk through the contents of the \texttt{stereo.default}
and discuss its various parameters.  If you want to start with a clean
slate, you can copy the \texttt{stereo.default.example} file that is
included in the top level of the Stereo Pipeline software distribution.

Note: The parameters that begin with `\texttt{DO\_*}' are true/false
options, when set to `1' they are `on' or `true,' and if set to `0'
they are `off' or `false.'  All parameters below have their default
values listed after the parameter name.

% -------------------------------------------------------------------
%                           PREPROCESSING
% -------------------------------------------------------------------

\section{Preprocessing}
\hrule
\bigskip

\begin{description}
\item[DO\_INTERESTPOINT\_ALIGNMENT \textnormal (default = 0)] \hfill \\
  When \texttt{DO\_INTERESTPOINT\_ALIGNMENT} is set to 1,
  \texttt{stereo} will attempt to pre-align the images by
  automatically detecting tie-points between images using a feature
  based image matching technique. Tiepoints are stored in a
  \texttt{*.match} file that is used to compute a linear affine
  transformation of the right image so that it closely matches the
  left image.  Note: the user may exercise more control over this
  process by using the \texttt{ipfind} and \texttt{ipmatch} tools.

\item[INTERESTPOINT\_ALIGNMENT\_SUBSAMPLING \textnormal{\small{(= 1,2,3,...,N)}} (default = 1)]\hfill \\
  {\em This settings is only for the ``keypoint'' stereo session.  It
    is not used for the ``isis'' stereo session.}

  Use this option to subsample images before interest point alignment
  when \\ DO\_INTERESTPOINT\_ALIGNMENT is activated.  This can
  significantly speed up the interest point alignment step at the
  expense of alignment accuracy.  When this is set to 1, there is no
  subsampling, and the \texttt{stereo} program will do its best to
  find as many interest points within the imagery as it can.  When
  this is set to $N > 1$, the program will subsample the images by a
  factor of $N$ before detecting interest points.  This parameter can
  be set to any positive integer.

\item[DO\_EPIPOLAR\_ALIGNMENT \textnormal (default = 0)] \hfill \\
  {\em Epipolar alignment is only available when performing stereo
    matches using the pinhole stereo session (i.e. when using}
    \texttt{stereo -t pinhole}){\em, and cannot be used when processing
    ISIS images at this time.}

  This method uses the inherent underlining geometry of the cameras to
  create a rectified version of the images where stereo disparity
  occurs only in the horizontal direction.

\item[FORCE\_USE\_ENTIRE\_RANGE \textnormal{\small{(= 0,1)}} (default = 0)] \hfill \\
  \emph{This setting is only for ISIS stereo session.}

  By default, the Stereo Pipeline will normalize ISIS images so that
  their maximum and minimum channel values are $\pm$2 standard
  deviations from a mean value of 1.0.  Use this option if you want to
  {\em disable} normalization in the stereo pipeline and force the raw
  values to pass directly to the stereo correlations algorithms.

  For example, if ISIS's \texttt{histeq} has already been used to
  normalize the images, then use this option to disable normalization
  as a (redundant) pre-processing step.


\item[DO\_INDIVIDUAL\_NORMALIZATION \textnormal (default = 0)] \hfill \\
  \emph{This setting is only for ISIS stereo session.}

  By default, the maximum and minimum valid pixel value is determined
  by looking at both images.  Normalized with the same ``global'' min
  and max guarantees that the two images will retain their brightness
  and contrast relative to each other.

  This option forces each image to be normalized to its own maximum
  and minimum valid pixel value. This is useful in the event that
  images have different and non-overlapping dynamic ranges. You can
  sometimes tell when this option is needed: after a failed stereo
  attempt one of the rectified images (\texttt{*-L.tif} and
  \texttt{*-R.tif}) may be either mostly white or black.  Activating
  this option may correct this problem.

  Note: Photometric calibration and image normalization are steps that
  can and should be carried out beforehand using ISIS's own utilities.
  This provides the best possible input to the stereo pipeline and
  yields the best stereo matching results.

\item[PREPROCESSING\_FILTER\_MODE \textnormal{\small{(= 0,1,2,3)}} (default = 3)] \hfill \\
  This selects the pre-processing filter to be used to prepare imagery
  before it is fed to the initialization stage of the pipeline.

  \begin{description}
    \item[0 - None]
    \item[1 - Gaussian Blur] - Pre-blur images using a Gaussian
      kernel.  This option can improve correlation results by blurring
      out small-scale image noise.
    \item[2 - LoG Filter] - Same as above, but take the Laplacian of
      the result.  This provides some immunity to differences in
      lighting conditions between a pair of images by isolating and
      matching on edge features in the image.
    \item[3 - Signed LoG Filter] - Same as above, but retain only the
      sign of the Laplacian image (+1 or -1).  This option provides
      the best immunity to variations in lighting conditions between
      images.
  \end{description}

  For modes 1, 2, and 3 above, the size of the Gaussian blur is
  determined by the PREPROCESSING\_KERNEL\_WIDTH variable below.

  The choice of pre-processing filter must be made with thought to the
  cost function being used (see COST\_MODE, below).  LoG filter preprocessing
  provides good immunity to variations in lighting conditions
  and is usually the recommended choice.  Signed LoG provides a speed
  advantage only when using the absolute difference cost function.
  Blurred preprocessing can sometimes produce the best results with
  well-calibrated images when working with the normalized cross
  correlation cost function.

\item[SLOG\_KERNEL\_WIDTH \textnormal{\small{(= \emph{float})}} (default = 1.5)] \hfill \\
  This defines the diameter of the Gaussian convolution kernel used
  for the preprocessing modes 1, 2, and 3 above. A value of 1.5 works
  well in a variety of applications.

\end{description}

% -------------------------------------------------------------------
%                           CORRELATION
% -------------------------------------------------------------------

\section{Correlation}
\hrule
\bigskip

\begin{description}

\item[COST\_MODE \textnormal{\small{(= 0,1,2)}}] (default = 2) \hfill \\
  The cost function used during initialization. As mention above,
  it is best to use absolute differences when working with the SLoG
  preprocessing filter. Otherwise, using the normalized cross
  correlation cost function is recommended.

  \begin{description}
    \item[0 - absolute difference]
    \item[1 - squared difference]
    \item[2 - normalized cross correlation]
  \end{description}

\item[COST\_BLUR \textnormal{\small{(= \emph{integer $N >= 0$})}} (default = 0)] \hfill \\
  Reduces the number of missing pixels by blurring the fitness
  landscape computed by the cost function by an $N \times N$ box filter.
  Increases the number of stereo matches during initialization at the
  expense of overall accuracy.  {\bf Cost blurring must be used in
    conjunction with affine adaptive window subpixel modes below,
    which are capable of achieving highly accurate results even when
    seeded by slightly inaccurate matches from the initialization
    step.}

\item[H\_KERNEL \textnormal{\small{(= \emph{integer})}} (default = 25)]
\item[V\_KERNEL \textnormal{\small{(= \emph{integer})}} (default = 25)] \hfill \\
  These two items determine the size (in pixels) of the correlation
  kernel used in the initialization step.  A different size can be set
  in the horizontal \emph{(H)} and vertical \emph{(V)} directions, but
  square correlation kernels are almost always used in practice.

\item[H\_CORR\_MIN \textnormal{\small{(= \emph{integer})}}]
\item[H\_CORR\_MAX \textnormal{\small{(= \emph{integer})}}]
\item[V\_CORR\_MIN \textnormal{\small{(= \emph{integer})}}]
\item[V\_CORR\_MAX \textnormal{\small{(= \emph{integer})}}] \hfill \\
  These parameters determine the size of the initial correlation
  search range.  The ideal search range depends on a variety of
  factors ranging from how the images were pre-aligned to the
  resolution and range of disparities seen in a given image pair.
  This search range is successively refined during initialization, so
  it is often acceptable to set a large search range that is guaranteed
  to contain all of the disparities in a given image.  However,
  setting tighter bounds on the search can sometimes reduce the number
  of erroneous matches, so it can be advantageous to tune the
  search range for a particular data set.

  Note: Commenting out these settings will cause \texttt{stereo} to make an
  attempt to guess its search range using interest points.

\item[SUBPIXEL\_MODE \textnormal{\small{(= 0,1,2,3)}} (default = 2] \hfill \\
  This parameter selects the subpixel correlation method. These
  algorithms are arranged in order of decreasing speed and increasing
  quality. Parabola subpixel is very fast but will produce results
  that are only slightly more accurate than those produced by the
  initialization step. Bayes EM (mode 2) is very slow but offers the
  best quality. When tuning {\tt stereo.default} parameters, it is
  expedient to start out using parabola subpixel as a ``draft mode.''
  When the results are looking good with parabola subpixel, then they
  will look even better with subpixel mode 3.

  \begin{description}
    \item[0 - no subpixel refinement]
    \item[1 - parabola fitting ]
    \item[2 - affine adaptive window, Bayes EM weighting ]
    \item[3 - affine adaptive window, Bayes EM with Gamma Noise Distribution (experimental) ]
  \end{description}

  For a visual comparison of the quality of these subpixel modes,
  refer back to Chapter:\ref{ch:correlation}.

\item[SUBPIXEL\_H\_KERNEL \textnormal{\small{(= \emph{integer})}} (default = 35)]
\item[SUBPIXEL\_V\_KERNEL \textnormal{\small{(= \emph{integer})}} (default = 35)] \hfill \\
  Specify the size of the horizontal \emph{(H)} and vertical
  \emph{(V)} size (in pixels) of the subpixel correlation kernel.

\end{description}

% -------------------------------------------------------------------
%                           FILTERING
% -------------------------------------------------------------------

\section{Filtering}
\hrule
\bigskip

\begin{description}

\item[FILL\_HOLES \textnormal{\small{(= 0,1)}} (default = 1)] \hfill \\
  When this option is on, the holes in the disparity map that result
  from poor stereo matching will be filled by an inpainting
  algorithm. Inpainting is a convolution method that takes the values
  at the edges of holes and spreads those values inward. This method
  performs best for small holes. The inpaint method is currently
  hard-coded so that it doesn't attempt to inpaint holes greater than
  100,000 pixels in total size.

  Note: you can always use the good pixel mask image ({\tt
    *-GoodPixelMap.TIF}) to determine which pixels represent ``real''
  data matched by the stereo correlator, and which pixels represent
  interpolated data produced by inpainting.

\item[RM\_H\_HALF\_KERN \textnormal{\small{(= \emph{integer})}} (default = 5)]
\item[RM\_V\_HALF\_KERN \textnormal{\small{(= \emph{integer})}} (default = 5)] \hfill \\
  Taken together, the RM\_* settings adjust the behavior of an outlier
  rejection scheme that ``erodes'' isolated regions of pixels in the
  disparity map that are in disagreement with their neighbors.

  The RM\_H\_HALF\_KERN and RM\_V\_HALF\_KERN parameters determine the
  size of the half kernel that is used to perform the automatic
  removal of low confidence pixels.  A $5 \times 5$ half kernel
  would result in an $11 \times 11$ kernel with 121 pixels in it.

\item[RM\_MIN\_MATCHES \textnormal{\small{(= \emph{integer})}} (default = 60)] \hfill \\
  This parameter sets the {\em percentage} of neighboring disparity
  values that must fall within the inlier threshold in order for a
  given disparity value to be retained.

\item[RM\_THRESHOLD \textnormal{\small{(= \emph{integer})}} (default = 3)] \hfill \\
  This parameter sets the inlier threshold for the outlier rejection
  scheme.  This option works in conjunction with RM\_MIN\_MATCHES
  above.  A disparity value is rejected if it differs by more than
  RM\_THRESHOLD disparity values from RM\_MIN\_MATCHES percent of
  pixels in the region being considered.

\item[RM\_CLEANUP\_PASSES \textnormal{\small{(= \emph{integer})}} (default = 1)] \hfill \\
  Select the number of outlier removal passes that are carried out.
  Each pass will erode pixels that do not match their neighbors.  One
  pass is usually sufficient.

\end{description}

% -------------------------------------------------------------------
%                           POST_PROCESSING
% -------------------------------------------------------------------

\section{Post-Processing}
\hrule
\bigskip

\begin{description}
\item[NEAR\_UNIVERSE\_RADIUS \textnormal{\small{(= \emph{float})}} (default = 0.0)]
\item[FAR\_UNIVERSE\_RADIUS \textnormal{\small{(= \emph{float})}} (default = 0.0)] \hfill \\
  These parameters can be used to filter out triangulated points in
  the 3D point cloud by setting an near and far radius value from
  origin of the point cloud's coordinate system, [0,0,0].  For most
  ISIS cameras, the origin is the center of the body (e.g. the Moon or
  Mars), and is part of a body-fixed Cartesian coordinate system that
  rotates with the planet.

  These settings are most useful for other stereo session types
  (e.g. pinhole), where the origin of the coordinate system is often
  one of the cameras in a stereo pair.  In this case, these parameters
  can be used to reject pixels that are too close or too far from the
  camera system.

  Setting both values zero turns off this restriction and allows the
  dot cloud to be as big as the data allows for.

\end{description}


