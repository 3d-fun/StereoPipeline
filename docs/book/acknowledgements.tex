\chapter*{Credits}

The Ames Stereo Pipeline was developed in the Intelligent Robotics
Group (IRG), a part of the Intelligent Systems Division at NASA Ames
Research Center in Moffett Field, CA.  It represents over ten years of
work leveraging IRG's extensive experience developing surface
reconstruction tools for terrestrial robotic field tests and planetary
exploration. \\ 

{\bf Lead Developer \& Project Lead:}
\begin {itemize} 
\item Michael J. Broxton (NASA/Carnegie Mellon University)\\ {\tt
  michael.broxton@nasa.gov})\\
\end{itemize}

{\bf Development Team:}
\begin{itemize}
\item Matthew Hancher (NASA),
\item Dr. Laurence Edwards (NASA)
\item Dr. Ross Beyer (NASA/SETI Institute)
\item Zachary Moratto (Kansas State University)
\item Dr. Ara Nefian (NASA/Carnegie Mellon University)
\item Mike Lundy (NASA/Stinger-Ghaffarian Technologies)\\ 
\item Vinh To (NASA/Stinger-Ghaffarian Technologies)
\end{itemize}

A number of student interns have made substantial contributions to the
Stereo Pipeline over the years: Sasha Aravkin (Washington State
University), Kyle Hussman (California Polytechnic Institute), Patrick
Mihelich (Stanford University), Melissa Bunte (Arizona State
University), Matthew Faulkner (Massachusetts Institute of Technology),
Todd Templeton (UC Berkeley), Morgon Kanter (Bard Colege), Kerri Cahoy
(Standford University), and Ian Saxton (UC San Diego).

This version of the Stereo Pipeline is based on an earlier version of
the Stereo Pipeline written by Eric Zbinden, Larry Edwards, and Judd
Bowman of the Intelligent Systems Division at NASA Ames Research
Center.

\section*{Acknowledgements}

The initial adaptation of the stereo pipeline to orbital imagers was a
result of a NASA project led by Dr. Michael Malin of Malin Space
Science Systems (MSSS) working in collaboration with Dr. Laurence
Edwards and Michael Broxton to develop automated Digital Elevation
Model generation techniques for the Mars Global Surveyor (MGS)
mission.  We thank Mike Malin, Mike Caplinger, and others at Malin
Space Science Systems for providing the seed funding that made the
current version of the Stereo Pipeline possible.

We'd also like to thank our friends and collaborators Randy Kirk,
Brent Archinal, Trent Hare, and Mark Roeiek of the USGS Astrology
Branch in Flagstaff, AZ for their encouragement and willingness to
share their experience and expertise by answering many of our
technical questions.  We also thank them for their ongoing effort to
help us evaluate our work.  Thanks also to the USGS ISIS team,
especially Jeff Anderson and Kris Becker, for their help in
intergrating the Stereo Pipeline with the USGS ISIS software package.

Thanks go also to Mark Robinon, Jacob Danton, Ernest Bowman-Cisneros,
Sam Laurence, and Melissa Bunte at Arizona State University for their
help in adapting the Stereo Pipeline to lunar data sets including the
Apollo Metric Camera.

Finally, we thank Melissa Bunte, Dr. Ara Nefian, and Larry Edwards and
for their contributions to this documentation, and Dr. Terry Fong (IRG
Group Lead) for his management and support of the open source \& public
software release process.

Portions of this software were developed with support from the
following NASA funding sources: the Mars Technology Program, the Mars
Critical Data Products Initiative, the Mars Reconnaissance Orbiter
mission, the Applied Information Systems Research program grant
\#06-AISRP06-0142, the Lunar Advanced Science and Exploration Research
(LASER) program grant \#07-LASER07-0148, and the ESMD Lunar Mapping and
Modeling Program.

\section*{Copyright and License}

This software and documentation is Copyright \copyright\ 2009 United
States Government as represented by the Administrator of the National
Aeronautics and Space Administration (NASA).  All Rights Reserved.

This software is distributed under the NASA Open Source Agreement
(NOSA), version 1.3.  The NOSA has been approved by the Open Source
Initiative.  See the file ``COPYING'' at the top of the distribution
directory tree for the complete NOSA document.

THE SUBJECT SOFTWARE IS PROVIDED "AS IS" WITHOUT ANY WARRANTY OF ANY
KIND, EITHER EXPRESSED, IMPLIED, OR STATUTORY, INCLUDING, BUT NOT
LIMITED TO, ANY WARRANTY THAT THE SUBJECT SOFTWARE WILL CONFORM TO
SPECIFICATIONS, ANY IMPLIED WARRANTIES OF MERCHANTABILITY, FITNESS FOR
A PARTICULAR PURPOSE, OR FREEDOM FROM INFRINGEMENT, ANY WARRANTY THAT
THE SUBJECT SOFTWARE WILL BE ERROR FREE, OR ANY WARRANTY THAT
DOCUMENTATION, IF PROVIDED, WILL CONFORM TO THE SUBJECT SOFTWARE.
