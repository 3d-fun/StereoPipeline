\chapter{Installation}

\section{Binary Installation}

This is the recommended method. Only two things are required:

\begin{description}
\item [{Stereo~Pipeline~Tarball.}] The main Stereo Pipeline page is \\
\url{http://ti.arc.nasa.gov/tech/asr/intelligent-robotics/ngt/stereo/}.
Download the \emph{Binary}
option that matches the platform you wish to use. The required \ac{ISIS}
version is listed next to the name; choose the newest version you
have available.

\item [{USGS~ISIS~Binary~Distribution.}] The Stereo Pipeline depends
on \ac{ISIS} version 3 from the \ac{USGS}\@. Their installation
guide is at \url{http://isis.astrogeology.usgs.gov/documents/InstallGuide}.
You must use their binaries as-is; if you need to recompile, you
must follow the \emph{Source Installation} guide for the Stereo
Pipeline in Section~\ref{sec:Source-Installation}.  Note also that
the \ac{USGS} provides only the current version of \ac{ISIS} and
the previous version (denoted with a `\texttt{\_OLD}' suffix) via
their \texttt{rsync} service.  If the current version is newer than
the version of ISIS that the Stereo Pipeline is compiled against,
be assured that we're working on rolling out a new version.  In the
meantime, you should be able to sync the previous version of ISIS
which should work with Stereo Pipeline.  To do so, view the listing
of modules that is provided via the `\texttt{rsync isisdist.wr.usgs.gov::}'
command.  You should see several modules listed with the `\texttt{\_OLD}'
suffix.  Select the one that is appropriate for your system, and
\texttt{rsync} according to the instructions.

If you have a need to keep current with ISIS, but don't want to
loose the ability to use Stereo Pipeline while we update our binaries
to the new current version of ISIS, you may wish to retain the
version of ISIS that matches your version of Stereo Pipeline.

\end{description}

\subsection{Quick Start}
\begin{description}

\item[{Fetch~Stereo~Pipeline}] ~\\
Download the Stereo Pipeline from \url{http://ti.arc.nasa.gov/stereopipeline}.

\item [{Fetch~ISIS~Binaries}] ~\\
\texttt{rsync -azv -\/-delete isisdist.wr.usgs.gov::isis3\_\textit{ARCH\_OS\_VERSION}/isis .}

\item [{Fetch~ISIS~Data}] ~\\
\texttt{rsync -azv --delete isisdist.wr.usgs.gov::isis3data/data/base data/}\\
\texttt{rsync -azv --delete isisdist.wr.usgs.gov::isis3data/data/\textit{MISSION} data/}

\item [{Untar~Stereo~Pipeline}] ~\\
\texttt{tar xzvf StereoPipeline-\textit{VERSION-ARCH-OS}.tar.gz}

% Verbatim has way too much white space. Couldn't seem to take care of it with
% vskip/vspace negative. Sigh.
\item [{Add~Stereo~Pipeline~to~Path~(optional)}] ~\\
bash: \texttt{export PATH="\textit{/path/to/StereoPipeline}/bin:\$\{PATH\}"} \\
csh:  \texttt{setenv PATH "\textit{/path/to/StereoPipeline}/bin:\$\{PATH\}"}

\item[Set~Up~Isis] ~\\
bash: \\
\hspace*{2em}\texttt{export ISISROOT=\textit{/path/to/isisroot}} \\
\hspace*{2em}\texttt{source \$ISISROOT/scripts/isis3Startup.sh} \\
csh: \\
\hspace*{2em}\texttt{setenv ISISROOT \textit{/path/to/isisroot}} \\
\hspace*{2em}\texttt{source \$ISISROOT/scripts/isis3Startup.csh}

\item [{Try~It~Out!}] ~\\
See the next chapter (Chapter~\ref{ch:tutorial}) for an example!
\end{description}

\subsection{Common Traps}

Here are some errors you might see, and what it could mean. Treat
these as templates for problems--in practice, the error messages might
be slightly different.

\begin{verbatim}
stereo: error while loading shared libraries: libisis3.so:
    cannot open shared object file: No such file or directory
\end{verbatim}

You just need to set up your ISIS environment.

\begin{verbatim}
dyld: Library not loaded: $ISISROOT/lib/libisis3.dylib
Referenced from: /some/path/goes/here/bin/program
Reason: image not found Trace/BPT trap
\end{verbatim}

You just need to set up your ISIS environment.

\begin{verbatim}
point2mesh E0201461-M0100115-PC.tif E0201461-M0100115-L.tif
[...]
99%  Vertices:   [************************************************************] Complete!
       > size: 82212 vertices
Drawing Triangle Strips
Attaching Texture Data
zsh: bus error  point2mesh E0201461-M0100115-PC.tif E0201461-M0100115-L.tif
\end{verbatim}

The source of this problem is an old version of OpenSceneGraph in
your library path. Check your \verb#LD_LIBRARY_PATH# (for Linux),
\verb#DYLD_LIBRARY_PATH# (for OSX), or your \verb#DYLD_FALLBACK_LIBRARY_PATH#
(for OSX) to see if you have an old version listed, and remove it
from the path if that is the case. It is not necessary to remove the
old versions from your computer, you just need to remove the reference
to them from your library path.

\pagebreak
\section{\label{sec:Source-Installation}Source Installation}

This method is for advanced users with moderate build system experience. Some dependencies such as ISIS and its dependencies \emph{(like SuperLU, Qwt, CSpice)} use custom build systems. Because of this and time, we won't help with questions on how to build dependencies.

\subsection{Dependency List}

This is a list of the direct dependencies of Stereo Pipeline. Some libraries
(like \ac{ISIS}) have dependencies of their own which are not covered here.

\begin{figure}[h]
  \centering
  \includegraphics[width=5in]{graph/asp_deps.pdf}
  \caption{Graph outlining some dependencies.}
\end{figure}

\begin{description}
\item [{Boost}] (Required) \url{http://www.boost.org/}\\
Version 1.35 or greater is required. Along with the base library
set, the Stereo Pipeline specifically requires: Program Options, Filesystem,
Thread, and Graph.

\item [{LAPACK}] (Required)\\
There are many sources for LAPACK\@. For OSX, you can use the
vecLib framework. For Linux, you can use the netlib LAPACK/CLAPACK
distributions, or Intel's MKL, or any of a number of others. The math
is unfortunately not a hotspot in the code, though, so using a faster
LAPACK implementation will not change much. Therefore, you should
probably just use the LAPACK your package manager (RPM for Red Hat
Linux, Yast for SuSE, etc.) has available.

\item [{ISIS}] (Recommended) \url{http://isis.astrogeology.usgs.gov/documents/InstallGuide}\\

The \ac{USGS} \acf{ISIS} library. This library handles the camera
models and image formats used for instruments.  \ac{ISIS} is usually
downloaded and used as a binary distribution.  Compilation of
\ac{ISIS} from source can be challenging, and their support forums may
provide assistance:
\url{https://isis.astrogeology.usgs.gov/IsisSupport/}. Cleaning and
modification of their source code maybe required if you would like to
use a newer version of ISIS's dependencies that may be available already
by your system.

\item [{OpenSceneGraph}] (Optional) \url{http://www.openscenegraph.org/}\\
OpenSceneGraph is required to run the \texttt{point2mesh} tool
(See Section~\ref{point2mesh}).

\item [{Vision~Workbench}] (Required) \url{http://ti.arc.nasa.gov/visionworkbench/}\\
Vision Workbench forms much of the core processing code of the
Stereo Pipeline.

\end{description}

\subsection{Build System}

The build system is built on GNU autotools. In-depth information on
autotools is available from \url{http://sources.redhat.com/autobook/}.
The basics, however, are simple. To compile the source code, first
run~\verb#./configure# from the top-level directory. This will search
for the dependencies and enable the modules you requested. There are
a number of options that can be passed to \verb#configure#; many
of these options can also be placed into a \verb#config.options#
file (in the form of \verb#VARIABLE="VALUE"#) in the same directory
as \verb#configure#. Table \ref{tab:Supported-configure-options}
lists the supported options.

\begin{table}
\begin{longtable}{|l|l|c|m{5cm}|}
\hline
\textbf{Variable Name}                & \textbf{Configure option}               & \textbf{Default} & \textbf{Function}\tabularnewline
\hline
\hline
\small\verb#PREFIX#                   & \small\verb#--prefix#                   & /usr/local       & Set the install prefix (ex: binaries will go in \$PREFIX/bin)\tabularnewline
\hline
\small\verb#HAVE_PKG_XXX#             & \small\verb#--with-xxx#                 & auto             & Set to {}``no'' to disable package XXX, or a path to only search that path\tabularnewline
\hline
\small\verb#PKG_PATHS#                & \small\verb#--with-pkg-paths#           & many             & Prepend to default list of search paths\tabularnewline
\hline
\small\verb#ENABLE_PKG_PATHS_DEFAULT# & \small\verb#--enable-pkg-paths-default# & yes              & Append built-in list of search paths\tabularnewline
\hline
\small\verb#ENABLE_OPTIMIZE#          & \small\verb#--enable-optimize#          & 3                & Level of compiler optimization?\tabularnewline
\hline
\small\verb#ENABLE_DEBUG#             & \small\verb#--enable-debug#             & no               & How much debug information?\tabularnewline
\hline
\small\verb#ENABLE_CCACHE#            & \small\verb#--enable-ccache#            & no               & Use \verb#ccache# if available\tabularnewline
\hline
\small\verb#ENABLE_RPATH#             & \small\verb#--enable-rpath#             & no               & Set RPATH on built binaries and libraries\tabularnewline
\hline
\small\verb#ENABLE_ARCH_LIBS#         & \small\verb#--enable-arch-libs#         & no               & Pass in 64 or 32 to look for libraries by default in \verb#lib64# or \verb#lib32#\tabularnewline
\hline
\small\verb#ENABLE_PROFILE#           & \small\verb#--enable-profile#           & no               & Use function profiling?\tabularnewline
\hline
\small\verb#PKG_XXX_CPPFLAGS#         &                                         &                  & Append value to CPPFLAGS for package XXX\tabularnewline
\hline
\small\verb#PKG_XXX_LDFLAGS#          &                                         &                  & Prepend value to LDFLAGS for package XXX\tabularnewline
\hline
\small\verb#PKG_XXX_LIBS#             &                                         &                  & Override the required libraries for package XXX\tabularnewline
\hline
\small\verb#PKG_XXX_MORE_LIBS#        &                                         &                  & Append to required libraries for package XXX\tabularnewline
\hline
\small\verb#ENABLE_EXCEPTIONS#        & \small\verb#--enable-exceptions#        & yes              & Use C++ exceptions? Disable at own risk.\tabularnewline
\hline
\small\verb#ENABLE_MULTI_ARCH#        & \small\verb#--enable-multi-arch#        & no               & OSX Only: Build \emph{Fat} binary with space-separated list of arches\tabularnewline
\hline
\small\verb#ENABLE_AS_NEEDED#         & \small\verb#--enable-as-needed#         & no               & Pass --as-needed to GNU linker. Use at your own risk.\tabularnewline
\hline
\end{longtable}\caption{\label{tab:Supported-configure-options}Supported configure options}
\end{table}
