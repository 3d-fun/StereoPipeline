\chapter{Installation}

\section{Binary Installation}

This is the recommended method. Only the Stereo Pipeline binaries
are required. ISIS is required only for users who wish to process
NASA non-terrestrial imagery.  A full ISIS installation is no longer
required for operation of Stereo Pipeline programs (only the ISIS
data directory is needed), but is required for certain preprocessing
steps before Stereo Pipeline programs are run for planetary data.
If you only want to process terrestrial Digital Globe imagery, skip
to section \ref{quickstartDG}.

\begin{description}
\item [{Stereo~Pipeline~Tarball.}] \hspace*{\fill} \\
The main Stereo Pipeline page is
\url{http://irg.arc.nasa.gov/ngt/stereo}.  Download the \emph{Binary}
option that matches the platform you wish to use. The recommend, but
optional, \ac{ISIS} version is listed next to the name; choose the
newest version you have available.

\item [{USGS~ISIS.}] \hspace*{\fill} \\
The Stereo Pipeline optionally depends on \ac{ISIS} version 3 from the
\ac{USGS}\@.  For processing planetary data, processing steps
with ISIS programs are needed prior to running Stereo Pipeline. 
However, this processing could be done on a completely separate
machine.  Stereo Pipeline itself uses ISIS internally, but the
Stereo Pipeline binaries now have a self-contained version of
ISIS (meaning that Stereo Pipeline itself doesn't depend on a
particular version of ISIS installed on your computer).  So 
while the Stereo Pipeline binary programs don't depend on a local 
ISIS installation, they will need access to an ISIS data directory
(if working with planetary data).

If you are working with planetary missions, you will need to install
ISIS so that you can perform preprocessing.  Their installation
guide is at \url{http://isis.astrogeology.usgs.gov/documents/InstallGuide}.
You must use their binaries as-is; if you need to recompile, you
must follow the \emph{Source Installation} guide for the Stereo
Pipeline in Section~\ref{sec:Source-Installation}.  Note also that
the \ac{USGS} provides only the current version of \ac{ISIS} and
the previous version (denoted with a `\texttt{\_OLD}' suffix) via
their \texttt{rsync} service.  If the current version is newer than
the version of ISIS that the Stereo Pipeline is compiled against,
be assured that we're working on rolling out a new version.  However,
since Stereo Pipeline has its own self-contained version of ISIS,
as long as there aren't major differences, you should be able to
use a slightly newer version of ISIS to preprocess your data, and
then still get good results with the Stereo Pipeline binaries.  If
not, you should be able to sync the previous version of ISIS which
should work with Stereo Pipeline.  To do so, view the listing of
modules that is provided via the
`\texttt{rsync~isisdist.astrogeology.usgs.gov::}' command.  You
should see several modules listed with the `\texttt{\_OLD}' suffix.
Select the one that is appropriate for your system, and \texttt{rsync}
according to the instructions.

The Stereo Pipeline should be able to work with data from newer
versions of ISIS than it was built against as long as the ISIS cube
format hasn't changed. Running the Stereo Pipeline executables only
requires that you have downloaded the ISIS secondary data and have
appropriately set the \texttt{ISIS3DATA} environment variable. This
is normally performed for the user by ISIS's startup script,
\texttt{\$ISISROOT/scripts/isis3Startup.sh}.

\end{description}

\subsection{Quick Start for ISIS users}
\begin{description}

\item[{Fetch~Stereo~Pipeline}] ~\\
Download the Stereo Pipeline from \url{http://irg.arc.nasa.gov/ngt/stereo}.

\item [{Fetch~ISIS~Binaries}] ~\\
As detailed at \url{http://isis.astrogeology.usgs.gov/documents/InstallGuide}.

\item [{Fetch~ISIS~Data}] ~\\
As detailed at \url{http://isis.astrogeology.usgs.gov/documents/InstallGuide}.

\item [{Untar~Stereo~Pipeline}] ~\\
\texttt{tar xzvf StereoPipeline-\textit{VERSION-ARCH-OS}.tar.gz}

% Verbatim has way too much white space. Couldn't seem to take care of it with
% vskip/vspace negative. Sigh.
\item [{Add~Stereo~Pipeline~to~Path~(optional)}] ~\\
bash: \texttt{export PATH="\textit{/path/to/StereoPipeline}/bin:\$\{PATH\}"} \\
csh:  \texttt{setenv PATH "\textit{/path/to/StereoPipeline}/bin:\$\{PATH\}"}

\item[Set~Up~ISIS] ~\\
bash: \\
\hspace*{2em}\texttt{export ISISROOT=\textit{/path/to/isisroot}} \\
\hspace*{2em}\texttt{source \$ISISROOT/scripts/isis3Startup.sh} \\
csh: \\
\hspace*{2em}\texttt{setenv ISISROOT \textit{/path/to/isisroot}} \\
\hspace*{2em}\texttt{source \$ISISROOT/scripts/isis3Startup.csh}

\item [{Try~It~Out!}] ~\\
See the next chapter (Chapter~\ref{ch:tutorial}) for an example!
\end{description}

\subsection{Quick Start for Digital Globe users}
\label{quickstartDG}
\begin{description}

\item[{Fetch~Stereo~Pipeline}] ~\\
Download the Stereo Pipeline from \url{http://irg.arc.nasa.gov/ngt/stereo}.

\item[{Untar~Stereo~Pipeline}] ~\\
\texttt{tar xvfz StereoPipeline-\textit{VERSION-ARCH-OS}.tar.gz}

\item [{Try~It~Out!}] ~\\
This documentation hasn't been finished yet. However
your imagery needs to be converted to the GeoTIFF format before ASP
can read it. ASP doesn't have native support for NITF that uses an
underlying JPEG2000 compression.

Once converted, the command line arguments to \texttt{stereo} look
like the following: ~\\
\texttt{stereo \textit{left-image} \textit{right-image} \textit{left-xml} \textit{right-xml} \textit{output-prefix}}

The settings discussed in the next chapter (Chapter~\ref{ch:tutorial})
apply to Digital Globe sessions as well. However the example requires
ISIS to be installed inorder to run. This may or may not be worth your
time to install.
\end{description}

\subsection{Common Traps}

Here are some errors you might see, and what it could mean. Treat
these as templates for problems.  In practice, the error messages might
be slightly different.

\begin{verbatim}
**I/O ERROR** Unable to open [$ISIS3DATA/Some/Path/Here].
Stereo step 0: Preprocessing failed
\end{verbatim}

You need to set up your ISIS environment or manually set the correct location for \texttt{ISIS3DATA}.

\begin{verbatim}
point2mesh E0201461-M0100115-PC.tif E0201461-M0100115-L.tif
[...]
99%  Vertices:   [************************************************************] Complete!
       > size: 82212 vertices
Drawing Triangle Strips
Attaching Texture Data
zsh: bus error  point2mesh E0201461-M0100115-PC.tif E0201461-M0100115-L.tif
\end{verbatim}

The source of this problem is an old version of OpenSceneGraph in
your library path. Check your \verb#LD_LIBRARY_PATH# (for Linux),
\verb#DYLD_LIBRARY_PATH# (for OSX), or your \verb#DYLD_FALLBACK_LIBRARY_PATH#
(for OSX) to see if you have an old version listed, and remove it
from the path if that is the case. It is not necessary to remove the
old versions from your computer, you just need to remove the reference
to them from your library path.

\newpage

\section{\label{sec:Source-Installation}Source Installation}

This method is for advanced users with moderate build system
experience. Some dependencies such as ISIS and its dependencies
\emph{(like SuperLU, Qwt, CSpice)} use their own custom build systems.
Due to the complex nature of the dependent software, we can't help you
with questions about those libraries.

In order to compile and build your own version of Stereo Pipeline you
will need the source code. The binary distribution that we provide
does not provide this. The source code for Stereo Pipeline is
available from Github at
\url{https://github.com/NeoGeographyToolkit/StereoPipeline}.

\subsection{Dependency List}

This is a list of the prime dependencies of Stereo Pipeline. Some libraries
(like \ac{ISIS} and \ac{VW}) have dependencies of their own which are not covered here.

\begin{figure}[h]
  \centering
  \includegraphics[width=5in]{graph/asp_deps.pdf}
  \caption{Graph outlining some dependencies. Not all of ISIS's are shown.}
\end{figure}

\begin{description}
\item [{Boost}] (Required) \url{http://www.boost.org/}\\
Version 1.46 or greater is required. Along with the base library
set, the Stereo Pipeline specifically requires: Program Options, Filesystem,
Thread, and Graph.

\item [{GDAL}] (Recommended) \url{http://www.gdal.org}\\
GDAL handles most of the File IO for Ames Stereo Pipeline. It also
provides support for the ingestion of proj4 strings from the
user. This is required if you wish to support the BigTIFF format and
write files larger that 4GB.

\item [{ISIS}] (Recommended) \url{http://isis.astrogeology.usgs.gov/documents/InstallGuide}\\
The \ac{USGS} \acf{ISIS} library. This library handles the camera
models and image formats used for instruments.  \ac{ISIS} is usually
downloaded and used as a binary distribution.  Compilation of
\ac{ISIS} from source can be challenging, and their support forums may
provide assistance:
\url{https://isis.astrogeology.usgs.gov/IsisSupport/}. Cleaning and
modification of their source code may be required if you would like to
use a newer version of ISIS's dependencies than may be available 
on your system.

ISIS is a complex suite of software to build from source, it is not
recommended that users try to build ISIS themselves.  Even though ISIS
provides pre-compiled libraries, not all of the headers are included.

\item [{LAPACK}] (Required)\\
There are many sources for LAPACK\@. For OSX, you can use the
vecLib framework. For Linux, you can use the netlib LAPACK/CLAPACK
distributions, or Intel's MKL, or any of a number of others. The math
is unfortunately not a hotspot in the code, though, so using a faster
LAPACK implementation will not change much. Therefore, you should
probably just use the LAPACK your package manager (RPM for Red Hat
Linux, Yast for SuSE, etc.) has available.

\item [{OpenSceneGraph}] (Optional) \url{http://www.openscenegraph.org/}\\
OpenSceneGraph is required to run the \texttt{point2mesh} tool (See
Section~\ref{point2mesh}). This library provides a convenient way of
building OpenGL graphics through the method of scene graphs. It also
provides a file format and utilities for display these scene
graphs. The output file of \texttt{point2mesh} is an OpenSceneGraph
binary scene graph format.

\item [{Python 2.4+}] (Required) \url{http://www.python.org}\\
Some applications of Stereo Pipeline are python scripts. Python
provides a friendly enviroment that hopefully encourages users to
attempt modifications of their own.

\item [{Vision~Workbench}] (Required) \url{http://ti.arc.nasa.gov/visionworkbench/}\\
Vision Workbench forms much of the core processing code of the Stereo
Pipeline. Vision Workbench contains almost all of the image processing
algorithms, such as image filters, image arithmetic, stereo
correlation, and triangulation. This means that Stereo Pipeline is
just a collection of applications that implement Vision Workbench in
the context of ISIS.

\end{description}

\subsection{Build System}

The build system is built on GNU autotools. In-depth information on
autotools is available from \url{http://sources.redhat.com/autobook/}.
The basics, however, are simple. To compile the source code, first
run~\verb#./configure# from the top-level directory. This will search
for the dependencies and enable the modules you requested. There are
a number of options that can be passed to \verb#configure#; many
of these options can also be placed into a \verb#config.options#
file (in the form of \verb#VARIABLE="VALUE"#) in the same directory
as \verb#configure#. Table \ref{tab:Supported-configure-options}
lists the supported options.

\begin{table}
\begin{longtable}{|l|l|c|m{5cm}|}
\hline
\textbf{Variable Name}                & \textbf{Configure option}               & \textbf{Default} & \textbf{Function}\tabularnewline
\hline
\hline
\small\verb#PREFIX#                   & \small\verb#--prefix#                   & /usr/local       & Set the install prefix (ex: binaries will go in \$PREFIX/bin)\tabularnewline
\hline
\small\verb#HAVE_PKG_XXX#             & \small\verb#--with-xxx#                 & auto             & Set to {}``no'' to disable package XXX, or a path to only search that path\tabularnewline
\hline
\small\verb#PKG_PATHS#                & \small\verb#--with-pkg-paths#           & many             & Prepend to default list of search paths\tabularnewline
\hline
\small\verb#ENABLE_PKG_PATHS_DEFAULT# & \small\verb#--enable-pkg-paths-default# & yes              & Append built-in list of search paths\tabularnewline
\hline
\small\verb#ENABLE_OPTIMIZE#          & \small\verb#--enable-optimize#          & 3                & Level of compiler optimization?\tabularnewline
\hline
\small\verb#ENABLE_DEBUG#             & \small\verb#--enable-debug#             & no               & How much debug information?\tabularnewline
\hline
\small\verb#ENABLE_CCACHE#            & \small\verb#--enable-ccache#            & no               & Use \verb#ccache# if available\tabularnewline
\hline
\small\verb#ENABLE_RPATH#             & \small\verb#--enable-rpath#             & no               & Set RPATH on built binaries and libraries\tabularnewline
\hline
\small\verb#ENABLE_ARCH_LIBS#         & \small\verb#--enable-arch-libs#         & no               & Pass in 64 or 32 to look for libraries by default in \verb#lib64# or \verb#lib32#\tabularnewline
\hline
\small\verb#ENABLE_PROFILE#           & \small\verb#--enable-profile#           & no               & Use function profiling?\tabularnewline
\hline
\small\verb#PKG_XXX_CPPFLAGS#         &                                         &                  & Append value to CPPFLAGS for package XXX\tabularnewline
\hline
\small\verb#PKG_XXX_LDFLAGS#          &                                         &                  & Prepend value to LDFLAGS for package XXX\tabularnewline
\hline
\small\verb#PKG_XXX_LIBS#             &                                         &                  & Override the required libraries for package XXX\tabularnewline
\hline
\small\verb#PKG_XXX_MORE_LIBS#        &                                         &                  & Append to required libraries for package XXX\tabularnewline
\hline
\small\verb#ENABLE_EXCEPTIONS#        & \small\verb#--enable-exceptions#        & yes              & Use C++ exceptions? Disable at own risk.\tabularnewline
\hline
\small\verb#ENABLE_MULTI_ARCH#        & \small\verb#--enable-multi-arch#        & no               & OSX Only: Build \emph{Fat} binary with space-separated list of arches\tabularnewline
\hline
\small\verb#ENABLE_AS_NEEDED#         & \small\verb#--enable-as-needed#         & no               & Pass --as-needed to GNU linker. Use at your own risk.\tabularnewline
\hline
\end{longtable}\caption{\label{tab:Supported-configure-options}Supported configure options}
\end{table}

\newpage
\section{\label{sec:Settings}Settings Optimization}

Finally the last thing to be done for Stereo Pipeline is to setup up
Vision Workbench's render settings. This step is optional, but for
best performance some thought should be applied here.

Vision Workbench is a multithreaded image processing library used by
Stero Pipeline. The settings by which Vision Workbench processes is
configurable by having a \texttt{.vwrc} file hidden in your home
directory. Below is an example.

\begin{minipage}{0.94\linewidth}
\small\listinginput{1}{LogConf.example}
\end{minipage}

There are a lot of possible options that can be implemented in the
above example. Let's cover the most important options and the concerns
the user should have when selecting a value.

\newpage
\begin{description}

\item[default\_num\_threads \textnormal (default=2)] \hfill \\
This sets the maximium number of threads that can be used for
rendering. When stereo's \texttt{subpixel\_rfne} is running you'll
probably notice 10 threads are running when you have
\texttt{default\_num\_threads} set to 8. This is not an error, you are
seeing 8 threads being used for rendering, 1 thread for holding
\texttt{main()}'s execution, and finally 1 optional thread acting as the
interface to the file driver.

It is usually best to set this parameter equal to the number of
processors on your system. Be sure to include the number of logical
processors in your arithmetic if your system supports hyper-threading.

Adding more threads for rasterization increases the memory demands of
Stereo Pipeline. If your system is memory limited, it might be best to
lower the \texttt{default\_num\_threads} option. Remember that 32 bit
systems can only allocate 4 GB of memory per process. Despite Stereo
Pipeline being a multithreaded application, it is still a single
process.

\item[write\_pool\_size \textnormal (default=21)] \hfill \\
The \texttt{write\_pool\_size} option represents the max waiting pool
size of tiles waiting to be written to disk. Most file formats do not
allow tiles to be written arbitrarily out of order. Most however will
let rows of tiles to be written out of order, while tiles inside a row
must be written in order. Because of the previous constraint, after a
tile is rasterized it might spend  some time waiting in the `write
pool' before it can be written to disk. If the `write pool' fills up,
only the next tile in order can be rasterized. That makes Stereo
Pipeline perform like it is only using a single processor.

Increasing the \texttt{write\_pool\_size} makes Stereo Pipeline more
able to use all processing cores in the system. Having this value too
large can mean excessive use of memory. For 32 bit systems again, they
can run out of memory if this value is too high for the same reason as
described for \texttt{default\_num\_threads}.

\item[system\_cache\_size \textnormal (default=805306368)] \hfill \\
Accessing a file from the hard drive can be very slow. It is especially
bad if an application needs to make multiple passes over an input
file. To increase performance, Vision Workbench will usually leave an
input file stored in memory for quick access. This file storage is
known as the 'system cache' and its max size is dictated by
\texttt{system\_cache\_size}. The default value is 768 MB.

Setting this value too high can cause your application to crash. It is
usually recommend to keep this value around 1/4 of the maximum
available memory on the system. For 32 bit systems, this means don't
set this value any greater than 1 GB. The units of this property is in
bytes.

\item[0 = *.progress] \hfill \\
This line is not assigning a value to progress, it is however setting
the logging level of progress bars. In the above example, this
statement is made under the \texttt{[logfile console]} state. This
means that only progress bars of type \texttt{ErrorMessage} will ever
be printed to the console. If you wanted progress bars up to type
\texttt{InfoMessage}, then the line in log file should be changed to:

\begin{verbatim}
    [logfile console]
    20 = *.progress
\end{verbatim}

\end{description}
