\chapter{Bundle Adjustment}

Bundle adjustment is the process of simultaneously adjusting the
properties of multiple cameras and the 3D locations of the objects
they see, to minimize the error between the estimated forward
projection of the 3D objects and their actual location in the captured
images. This has application as an optional step between the capture
of images and the creation of DEMs. There is always an amount of error
with the recorded position and orientation of cameras and Bundle
Adjustment can be used to refine these measurements. This will allow
DEMs from multiple cameras to align better with one another. Bundle
Adjustment can also take advantage of Ground Control Points (GCPs),
which are accurately measured 3D locations on the surface of DEM. GCPs
can be used to improve the alignment of the DEMs or align the new DEM
to a past data product. Finally, even though Bundle Adjustment
calculates the location of the 3D objects it views, only the final
properties of cameras are recorded in the Ames Stereo Pipeline.

\subsection{A deeper understanding}

Bundle Adjustment in Ames Stereo Pipeline revolves around the
Levenberg-Marquardt Algorithm (LMA) which is a method of minimizing an
error. In the case of Bundle Adjustment, the error is the pixel
difference between an objects location in an image and it's forward
projection through the camera. The goal is to solve for an update to
the parameters of the cameras and point positions, apply them, and
then repeat until the update supplied by LMA goes to zero.

LMA is a hybrid of two minimization techniques, gauss-newton and
gradient decent. Where the control between these two methods is the
parameter lambda which will change in value during the process of
Bundle Adjustment. A high value of lambda forces LMA to act like
gradient decent, and is what will happen at the beginning of bundle
adjustment when the camera parameters are far away from their final
solution. A low value of lambda drives LMA into the gauss-newton
method; at which it will take small steps in updating the camera
values. When Bundle Adjustment is almost finished and is close to the
solution, it will lower lambda.

Since Bundle Adjustment is an iterative method, it would be happy to
keep processing new updates to camera parameters all day. To avoid
this, there are several shutoff conditions. The first is when the
update to the camera parameters becomes insignificantly small. The
second is when the error measurement becomes insignificantly
small. Both of these conditions' thresholds are defined withing the
bundle adjuster's code and is not allowed for the user to change. The
final shut off condition is when the number of iterations becomes too
large. It is important to understand that when this shut off condition
happens, bundle adjustment has not finished refining the parameters of
the cameras but they are a step closer to the solution. The maximum
number of iterations is changeable by the user so they can decide how
much time they're will to dedicate to the correction of the data. The
number of iterations Bundle Adjustment takes to converge on the
solution can be anywhere between 20 iterations or several hundred.

If you are interested in more information on the math of Bundle
Adjustment and the arrangement of the problem, we recommend reading
Appendix 6 in the {\em Multiple View Geometry book} \cite{hartley04}. 
For more information on why LMA is used instead of the million other 
optimization algorithms, try reading {\em Bundle Adjustment – A Modern 
Synthesis} \cite{triggs00}.

\section{RMAX Adjust}

\texttt{rmax\_adjust} is a specific bundle adjuster for data collected by IRG on
the Yamaha RMAX Helicopter. It is also our testing grounds to
determine if any changes to Bundle Adjustment code works.

\subsection{Options}

\begin{verbatim}
--cnet, -c [control network file]
\end{verbatim}

\emph{Optional.} Feeding this option will force RMAX Adjust to use an
already built control network. If not fed, RMAX Adjust will look for
match files in current operating directory with names similar to the
input images to build it's control network from. If \texttt{rmax\_adjust}
creates it's own control network file, it will save it as
\verb=rmax_adjust.cnet=.

\begin{verbatim}
--lambda, -l [value]
\end{verbatim}

\emph{Optional.} This set the starting value for lambda. Bundle
Adjustment will naturally select a value for lambda it thinks is best
for the starting error, but this argument can be used to override
that.

\begin{verbatim}
--min-matches [default = 5]
\end{verbatim}

When producing a control network, this sets the minimum required
interest point matches between images for them to be included. If a
match file fails to find this many of matches, it probably means these
were poor matches and the images don't really overlap.

\begin{verbatim}
--max-iterations [default = 25]
\end{verbatim}

Sets the maximum number of iterations to be done by Bundle Adjustment.

\begin{verbatim}
--nonsparse, -n
\end{verbatim}

\emph{Optional.} Switches the Bundle Adjustment code to use non-sparse
matrices in it's math. This will cause it to run significantly slower
and is only used as a check of the programmer's sparse matrix code.

\begin{verbatim}
--report-level, -r [default = 10]
\end{verbatim}

Sets the report level for the final report on the bundle adjustment
which can be found in \verb=rmax_adjust.report=. Report levels are defined in
BundleAdjustReport.h in Vision Workbench.

\begin{verbatim}
--run-match, -m
\end{verbatim}

\emph{Optional.} If match files don't already exist, create them using a
call to \texttt{IPmatch}.

\begin{verbatim}
--match-debug-images, -d
\end{verbatim}

\emph{Optional.} If a call to \texttt{IPmatch} is being called, this option
also allows for the creation of debugging images that show the matches
between the input images.

\subsection{Examples of Use}

--> steal from my old report?

\section{ISIS Adjust}

Purpose, Bundle Adjustment Model, Supported Equations, Examples

\section{Visualizing Bundle Adjustment with BundleVis}

Purpose, Examples, Supported Cameras, and File Types

\begin{thebibliography}{1}

\bibitem{hartley04} Hartley, R.I. and Zisserman, A. ``Multiple View Geometry in Computer Vision,''
  Cambridge University Press. 2004. pp 597-627.
\bibitem{triggs00} Triggs, McLauchlan, Hartley, and Fitzgibbon. ``Bundle Adjustment - A Modern Synthesis,''
  Lecture Notes in Computer Science. Vol. 1883, 298. January 2000

\end{thebibliography}
