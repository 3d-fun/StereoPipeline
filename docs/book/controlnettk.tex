\chapter{Control Network Toolkit}
\label{ch:controlnettk}

This chapter covers the first additional boost of tools for \ac{ASP}!
Control Networks are a data structure that at it's core contains image
features that can be tracked in multiple images. Since these features
can be tracked in multiple images, they can be triangulated and
represent a 3D location. This control network can then be used in
processes such as Bundle Adjustment to refine camera locations. These
refined locations allows for seamless mosaics for a clear view of the
subject.

\emph{Warning: This toolkit is not finished but this documentation
  serves as a place holder for an Official release.}

So .. I'm not too excited about writing documentation at this hour but
I'm willing to provide some examples. The method of developing a
control network with \ac{VW} and \ac{ASP} is that we try to track as
many 'natural' features and then reduce into a control network. This
is done in a 6 step process.

\begin{itemize}
\item Detect features using a second order filter.
\item Describe these features by their surroundings with a tag.
\item Match tags of interest points between images.
\item Filter these matches for error using RANSAC.
\item Reduce matches further for ease of processing and also to assure uniform distribution.
\item Collect all pair-wise matches and write as Control Network.
\end{itemize}

I'm going to provide with an example of how to use this software with
Apollo Metric images. Hopefully you can follow along with whatever
imagery you happen to have around.

We are going to start out first by gathering interest points \emph{(or
  features)} and this is accomplished with a tool called
\texttt{ipfind}. This handy tool is provided by \ac{VW} and is not
built against ISIS. This means it can only read cube files through a
library called GDAL. To make sure cubes files can be read correctly by
GDAL and thus ipfind, we'll have to convert our input imagery to
something more reliable like TIFFs.

\begin{verbatim}
    ISIS 3> isis2std from=INPUT to=OUTPUT.tif format=TIFF
\end{verbatim}

I don't like to call isis2std on every file myself. Here's the
commands I use to do this in parallel.

\begin{verbatim}
    ISIS 3> echo *.cub | xargs -n1 echo |
              awk -F "." '{print "from="$1".cub to="$1".tif format=TIFF"}' |
              xargs -n3 -P 10 isis2std
\end{verbatim}

The directory should now be filled with TIFF format doppelgangers. We
are ready to perform \texttt{ipfind} which will detect features and
describe them in one go. The results of each operation will be saved
in a \texttt{.vwip} file which will later be read during
matching. There are many algorithms that \texttt{ipfind} can use for
detection and description but the defaults of OBALoG and SGrad should
do fine for most situations. Also note that \texttt{ipfind} has a
debug flag '-d' which can be used to output debug images that show the
location of all detected features.

\begin{verbatim}
    ASP   > echo *.tif | xargs -n1 -P 10 ipfind
\end{verbatim}

Matching is calculated pairwise. There are many ways to do this, but
the simplists is just a brute force permutation of all possible
combinations. Here's how:

\begin{verbatim}
    ASP   > pairlist_all *.tif | xargs -n2 -P 10 ipmatch -r homography
\end{verbatim}

Though for large datasets it doesn't seem approriate to compare all
images to each other as the physically don't see each other in
anyway. A good way of reducing the matches is by deciding to only do
match between images that are only seperated by a few
degrees. \emph{This code may or may not exist yet. I wrote down a mock
  up of what I'd like to happen.}

\begin{verbatim}
    ASP   > pairlist_degree *cub -a 10 -iextension tif |
              xargs -n2 -P 10 ipmatch -r homography
\end{verbatim}

If you've been reading the output from ipmatch, you may have noticed
that some pair wise matches might have more than a 100 matches! This
is probably overkill for some cases and could potentially chock an
application further down the process. Also at this point, or interest
points could be located anywhere. The worst case is that all matched
feature have managed to clump around interesting objects like a
sparkly crater or a sharp cut rill. We can't enforce that the matched
features are evenly placed, but we can trim them down to be somewhat
even.

Enter stage left, \texttt{reduce\_match}.
