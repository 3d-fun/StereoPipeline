
\chapter*{Credits}

This open source version of the \ac{ASP} was developed by the
\ac{IRG}, in the Intelligent Systems Division at the \ac{NASA} Ames
Research Center in Moffett Field, CA. It builds on over ten years
of IRG experience developing surface reconstruction tools for
terrestrial robotic field tests and planetary exploration. \\

{\bf Project Lead}
\begin {itemize}
\item Zachary Moratto (NASA/Stinger-Ghaffarian Technologies)\\ {\tt z.m.moratto@nasa.gov}
\end{itemize}

{\bf Development Team}
\begin{itemize}
\item Oleg Alexandrov (NASA/Stinger-Ghaffarian Technologies)
\item Scott McMichael (NASA/Stinger-Ghaffarian Technologies)
\item Dr.~Ross Beyer (NASA/SETI Institute)
\end{itemize}

{\bf Former Developers}
\begin{itemize}
\item Dr.~Ara Nefian (NASA/Carnegie Mellon University)
\item Matthew Hancher (NASA)
\item Mike Lundy (NASA/Stinger-Ghaffarian Technologies)
\item Vinh To (NASA/Stinger-Ghaffarian Technologies)
\end{itemize}

{\bf Contributing Developer \& Former Principal Investigator (LMMP)}
\begin{itemize}
\item Michael J.~Broxton (NASA/Carnegie Mellon University)
\end{itemize}

{\bf Contributing Developer \& Former IRG Terrain Reconstruction Lead}
\begin{itemize}
\item Dr.\ Laurence Edwards (NASA)
\end{itemize}

A number of student interns have made significant contributions to
this project over the years: Kyle Husmann (California Polytechnic
State University), Sasha Aravkin (Washington State University),
Aleksandr Segal (Stanford), Patrick Mihelich (Stanford University),
Melissa Bunte (Arizona State University), Matthew Faulkner
(Massachusetts Institute of Technology), Todd Templeton (UC Berkeley),
Morgon Kanter (Bard College), Kerri Cahoy (Stanford University), and
Ian Saxton (UC San Diego).

The open source Stereo Pipeline leverages stereo image processing
work, past and present, led by Michael J. Broxton (NASA/CMU),
Dr. Laurence Edwards (NASA), Eric Zbinden (formerly NASA/QSS Inc.),
Dr.~Michael Sims (NASA), and others in the Intelligent Systems
Division at NASA Ames Research Center. It has benefited substantially
from the contributions of Dr.~Keith Nishihara (formerly
NASA/Stanford), Randy Sargent (NASA/Carnegie Mellon University),
Dr.~Judd Bowman (formerly NASA/QSS Inc.), Clay Kunz (formerly NASA/QSS
Inc.), and Dr.~Matthew Deans (NASA).

\section*{Acknowledgments}

The initial adaptation of Ames' stereo surface reconstruction tools to
orbital imagers was a result of a NASA funded, industry led project to
develop automated \ac{DEM} generation techniques for
the \ac{MGS} mission. Our work with that project's
Principal Investigator, Dr.~Michael Malin of Malin Space Science
Systems (MSSS), and Co-Investigator, Dr.~Laurence Edwards of NASA
Ames, inspired the idea of making stereo surface reconstruction
technology available and accessible to a broader community.  We thank
Dr.~Malin and Dr.~Edwards for providing the initial impetus that in no
small way made this open source stereo pipeline possible, and we thank
Dr.~Michael Caplinger, Joe Fahle and others at MSSS for their help and
technical assistance.

We'd also like to thank our friends and collaborators Dr.~Randolph
Kirk, Dr.~Brent Archinal, Trent Hare, and Mark Rosiek of the
\aclu{USGS}'s (\acs{USGS}'s) Astrogeology Science Center in Flagstaff,
AZ, for their encouragement and willingness to share their experience
and expertise by answering many of our technical questions.  We also
thank them for their ongoing support and efforts to help us evaluate
our work.  Thanks also to the \ac{USGS} \ac{ISIS} team, especially
Jeff Anderson and Kris Becker, for their help in integrating stereo
pipeline with the \ac{USGS} \ac{ISIS} software package.

Thanks go also to Dr.~Mark Robinson, Jacob Danton, Ernest
Bowman-Cisneros, Dr.~Sam Laurence, and Melissa Bunte at Arizona State
University for their help in adapting the Ames Stereo Pipeline to
lunar data sets including the Apollo Metric Camera.

We'd also like to thank David Shean, Dr.~Ben Smith, and Dr.~Ian
Joughin of the Applied Physics Laboratory at the University of
Washington for providing design direction for adapting Ames Stereo
Pipeline to Earth sciences and in particular the Digital Globe mode.

Finally, we thank Dr.~Ara Nefian, and Dr.~Laurence Edwards for their
contributions to this documentation, and Dr.~Terry Fong (IRG Group
Lead) for his management and support of the open source and public
software release process.

Portions of this software were developed with support from the
following NASA Science Mission Directorate (SMD) and Exploration
Systems Mission Directorate (ESMD) funding sources: the Mars
Technology Program, the Mars Critical Data Products Initiative, the
Mars Reconnaissance Orbiter mission, the Applied Information Systems
Research program grant \#06-AISRP06-0142, the Lunar Advanced Science
and Exploration Research (LASER) program grants \#07-LASER07-0148 and
\#11-LASER11-0112, the ESMD Lunar Mapping and Modeling Program (LMMP),
and the SMD Cryosphere Program.

Any opinions, findings, and conclusions or recommendations expressed
in this documentation are those of the authors and do not necessarily
reflect the views of the National Aeronautics and Space Administration.
