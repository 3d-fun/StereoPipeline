\chapter{Programs}

This chapter covers the various user-programs that are a part of
the Ames Stereo Pipeline.

\section{stereo}

The \texttt{stereo} program is the primary workhorse of the Ames
Stereo Pipeline.  It is the program that takes a pair of images which
overlap and creates an output point cloud which can then be fed to the
\texttt{point2mesh} or \texttt{point2dem} programs.

\emph{Way more needs to go in here ...}

\subsection{stereo.default file}

The \texttt{stereo.default} file contains configuration parameters
which the \texttt{stereo} program uses to process the images.  Below
we will walk through the contents of the \texttt{stereo.default.example}
file distributed with the Ames Stereo Pipeline and discuss all of
the various parameters.

The parameters which begin with `\texttt{DO\_}' are true/false options,
when set to `1' they are `on' or `true,' and if set to `0' they are
`off' or `false.'

The parameters below also have their default values listed after
the parameter name.

\subsubsection*{Preprocessing}

\begin{description}
\item[DO\_INTERESTPOINT\_ALIGNMENT 1] \hfill \\
When \texttt{DO\_INTERESTPOINT\_ALIGNMENT} is on (or set to 1), ... \emph{MJB: describe}

\item[DO\_EPIPOLAR\_ALIGNMENT 0] \hfill \\
By default this is off.  When on ... \emph{MJB: describe}

\item[INTERESTPOINT\_ALIGNMENT\_SUBSAMPLING 1] \hfill \\
This option allows you to change the density of interest points
that stereo will find, correlate, and result in the final point
cloud.  When this is set to 1, there is no subsampling, the
\texttt{stereo} program will do its best to find as many interest
points within the imagery as it can.  When this is set to 2, the
program will ignore every other interest point that it finds, and
will only process the reduced set.  This parameter can be set to
any positive integer.  When this parameter is turned up, the resulting
point cloud will have less effective resolution.

\item[DO\_SLOG 1]
\item[DO\_LOG 0] \hfill \\
These two items are related, only one can be set to `on', if both
are `on' the program will default to doing only SLOG.  \emph{MJB: explain the difference}

\item[SLOG\_KERNEL\_WIDTH 1.5] \hfill \\
When \texttt{DO\_SLOG} is `on,' this option sets the diameter of
the convolution kernel. \emph{MJB: describe}

\end{description}

\subsubsection*{Correlation}

\begin{description}
\item[H\_KERNEL 25]
\item[V\_KERNEL 25] \hfill \\
These two items determine the size of the kernel in the horizontal (H) and vertical (V) directions in the input images.  \emph{MJB: more detail}

\item[SUBPIXEL\_H\_KERNEL 25]
\item[SUBPIXEL\_V\_KERNEL 25]
These two items are only relevant when the \texttt{DO\_H\_SUBPIXEL} and \texttt{DO\_V\_SUBPIXEL} parameters (detailed below) are `on.'  They speciffy the size of the subpixel kernel in the horizontal (H) and vertical (V) directions.

\item[H\_CORR\_MIN -100]
\item[H\_CORR\_MAX 100]
\item[V\_CORR\_MIN -100]
\item[V\_CORR\_MAX 100] \hfill \\
These parameters determine the size of the correlation window that the kernel will be moved around within to find a match.

\item[SUBPIXEL_MODE 0] \hfill \\
This parameter determines the method by which subpixel correlation is done



\end{description}

\subsubsection*{Filtering}
\subsubsection*{Dot Cloud}

\section{disparitydebug}
\section{results}
\section{point2mesh}
\section{point2dem}
\section{orthoproject}
