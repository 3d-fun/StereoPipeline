\chapter{Guide to Output Files}
\label{chapter:outputfiles}

The {\tt stereo} tool generates a variety of intermediate files that
are useful for debugging.  These are listed below, along with brief
descriptions about the contents of each file.  Note that the prefix of
the filename for all of these files is dictated by the final command
line argument to {\tt stereo}.  Run {\tt stereo -\/-help} for details.

\begin{description}

\item[*.vwip \textnormal{- image feature files}] \hfill \\
  If \texttt{alignment-method homography} is enabled, the stereo
  pipeline will automatically search for image features to use for
  tie-points.  Raw image features are stored in \texttt{*.vwip} files;
  one per input image. For example, if your images are
  \texttt{left.cub} and \texttt{right.cub} you'll get
  \texttt{left.vwip} and \texttt{right.vwip}.  Note: these files can
  also be generated by hand (and with finer grained control over
  detection algorithm options) using the {\tt ipfind} utility.

\item[*.match \textnormal{- image to image tie-points}] \hfill \\ 
  The match file lists a select group of unique points out of the
  previous \texttt{.vwip} files that have been identified and matched
  in a pair of images.  For example, if your images are
  \texttt{left.cub} and \texttt{right.cub} you'll get a
  \texttt{left\_\_right.match} file.  

  The \texttt{.vwip} and \texttt{.match} files are meant to serve
  as cached tie-point information, and they help speed up the
  pre-processing phase of the stereo pipeline: if these files exist
  then the \texttt{stereo} program will skip over the interest point
  alignment stage and instead use the cached tie-points contained
  in the \texttt{*.match} files.  In the rare case that one of these files
  did get corrupted or your input images have changed, you may want
  to delete these files and allow {\tt stereo} to regenerate them
  automatically.  This is also recommended if you have upgraded the
  Stereo Pipeline software.

\item[*-L.tif - \textnormal{rectified left input image}] \hfill \\ 
  The left input image of the stereo pair, saved after the
  pre-processing step.  This image may be normalized, but should
  otherwise be identical to the original left input image.

\item[*-R.tif - \textnormal{rectified right input image}] \hfill \\
  Right input image of the stereo pair, after the pre-processing
  step.  This image may be normalized and possibly
    translated, scaled, and/or rotated to roughly align it with the left
    image, but should otherwise be identical to the original right
  input image.

\item[*-lMask.tif \textnormal{- mask for left rectified image}]
\item[*-rMask.tif \textnormal{- mask for right rectified image}] \hfill \\
  These files contain binary masks for the input images.  These are
  used throughout the stereo process to mask out pixels where there is
  no input data.

\item[*-align.exr \textnormal{- pre-alignment matrix}] \hfill \\
  The $3 \times 3$ affine transformation matrix that was used to warp the right
  image to roughly align with the left image.  This file is only
  generated if \texttt{alignment-method homography} is enabled in the
  {\tt stereo.default} file.

\item[*-D.tif \textnormal{- disparity map after the disparity map initialization phase}] \hfill \\
  This is the disparity map generated by the correlation algorithm in
  the initialization phase.  It contains integer values of disparity
  that are used to seed the subsequent sub-pixel correlation phase.
  It is largely unfiltered, and may contain some bad matches.

  Disparity map files are stored in OpenEXR format as 3-channel,
  32-bit floating point images.  (Channel 0 = horizontal disparity,
  Channel 1 = vertical disparity, and Channel 2 = good pixel mask)

\item[*-RD.tif - \textnormal{disparity map after sub-pixel correlation}] \hfill \\
  This file contains the disparity map after sub-pixel refinement.
  Pixel values now have sub-pixel precision, and some outliers have
  been rejected by the sub-pixel matching process.

\item[*-F-corrected.tif \textnormal{- intermediate data product}] \hfill \\
  Only created when \texttt{alignment-method homography} is on.
  This is \texttt{*-F.tif} with effects of interest point alignment removed.

\item[*-F.tif \textnormal{- filtered disparity map}] \hfill \\
  The filtered, sub-pixel disparity map with outliers removed (and
  holes filled with the inpainting algorithm if \texttt{FILL\_HOLES}
  is on). This is the final version of the disparity map.

\item[*-GoodPixelMap.tif \textnormal{- map of good pixels}] \hfill \\
  An image showing which pixels were matched by the stereo correlator
  (gray pixels), and which were filled in by the hole filling
  algorithm (red pixels).

\item[*-PC.tif \textnormal{- point cloud image}] \hfill \\
  The point cloud image is generated by the triangulation phase of the
  Stereo Pipeline.  It contains 3D locations for each valid pixel;
  stored as a 64-bit, 3-channel TIFF, with coordinates in a body-fixed
  planetocentric coordinate system.  Each pixel in the point cloud
  image corresponds to a pixel in the left input image.

  Note: it is unlikely that your usual TIFF viewing programs will
  visualize this file properly.  This file should be considered a
  `data' file, not an `image' file.  Other programs in the Stereo
  Pipeline, such as {\tt point2mesh} and {\tt point2dem} will convert
  the contents of this file to more easily visualized formats.

\item[*-stereo.default \textnormal{- backup of the stereo pipeline settings file}] \hfill \\ 
  This is a copy of the \texttt{stereo.default} file used by
  \texttt{stereo}.  It is stored alongside the output products as
  a record of the settings that were used for this particular stereo
  processing task.

\end{description}
