\chapter{Stereo Pipeline Output Files}

The files that \verb=stereo= creates are as follows (assuming
\verb=<output_file_prefix>= is set to be `\verb=out=' although it need
not be):

\begin{description}

\item[VWIP extension] \hfill \\ 
  One of these Vision Workbench Interest
  Point files will be created for each input image. If your images are
  \texttt{left.cub} and \texttt{right.cub} you'll get
  \texttt{left.vwip} and \texttt{right.vwip}. The \texttt{.vwip} file
  is an intermediate file that highlights unique points in an
  image. They are created with the default settings of \texttt{ipfind}
  from the Vision Workbench.

\item[MATCH extension] \hfill \\ 
  The match file lists a select group
  of unique points out of the previous \texttt{.vwip} files that have
  been properly identified in both images. This match file is used to
  calculate the intial overlap of the images if
  \texttt{DO\_INTERESTPOINT\_ALIGNMENT} has been set.  \emph{Again, if
    your images are \texttt{left.cub} and \texttt{right.cub} you'll
    get a \texttt{left\_\_right.match} file.}

  The \texttt{.vwip} and \texttt{.match} files will only be created if
  \texttt{DO\_INTERESTPOINT\_ALIGNMENT} has been set.  These files can
  help speed up the process while you fool around with parameters in
  your \texttt{stereo.default} file, since their contents are not
  affected by other parameters.  If these files exist, then the
  \texttt{stereo} program will skip over the interest point alignment
  stage and just use these files.  So if you run \texttt{stereo} on
  the same pair of images more than once, don't delete these files,
  take advantage of them.  Of course, if you do, \texttt{stereo} will
  just rebuild them.  Also, you don't have to worry about these files
  being `bad' since their contents aren't affected by changes to the
  \texttt{stereo.default} file.  In the rare case that one of these
  files did get corrupted, you would get a `failed to read interest
  point file' error message from \texttt{stereo}.

\item[*-L.TIF suffix] \hfill \\
  Left image in the stereo pair (mostly identical to input).

\item[*-R.TIF suffix] \hfill \\
  Right (ailgned) image of the stereo pair.

\item[*-lMask.TIF suffix]
\item[*-rMask.TIF suffix] \hfill \\
  Image masks

\item[*-align.EXR suffix] \hfill \\
  Alignment.  The linear affine transformation that
  results from interest point alignment.  If
  \texttt{DO\_INTERESTPOINT\_ALIGNMENT} is off in the
  \texttt{stereo.default} file, then this file won't get made.

\item[*-D.EXR suffix] \hfill \\
  This is the unfiltered disparity map, straight
  out of the integer pixel correlator.  Contains only integer values
  of disparity.

\item[*-R.EXR suffix] \hfill \\
  Disparity map after sub-pixel refinement.

\item[*-F-corrected.EXR suffix] \hfill \\
  Only created when \texttt{DO\_INTERESTPOINT\_ALIGNMENT} is on.
  Disparity map with effects of interest point alignment removed.
  Intermediate data product.

\item[*-F.EXR suffix] \hfill \\
  The Filtered, sub-pixel disparity map with outlier removal and holes
  filled in (only exists if \texttt{FILL\_HOLES\_NURBS} is on).

\item[*-GoodPixelMap.TIF suffix] \hfill \\
  An image showing which pixels were matched by the stereo correlator,
  and which were filled in by the hole filling algorithm.

\item[*-PC.TIF suffix] \hfill \\
  Point Cloud image with 3D locations for each point.  Stored as a
  64 bit, 3 channel TIFF, with coordinates in body-fixed planetocentric
  coordinate system.  Odds are good that your usual TIFF viewer
  programs will not visualize this file properly.  This file should
  really be considered a `data' file, not an `image' file.  Other
  programs in the Stereo Pipeline will convert the contents of this
  file to more easily vizualized formats.

\item[*-stereo.default file] \hfill \\
  This is a copy of the \texttt{stereo.default} file
  used during the run of \texttt{stereo} that built all of these
  files.

\end{description}

