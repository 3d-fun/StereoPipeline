\chapter{Introduction}

The NASA Ames Stereo Pipeline (ASP) is an automated stereogrammetry
tool designed for processing planetary imagery captured from orbiting
and landed robotic explorers on other planets.  It was designed to
process stereo imagery captured by NASA spacecraft and produce
cartographic products including digital elevation models (DEMs),
ortho-projected imagery, and 3D models.  These data products are
suitable for science analysis, mission planning, and public outreach.

\section{Background}

The Intelligent Robotics Group (IRG) at the NASA Ames Research Center
(ARC) has been developing 3D surface reconstruction and visualization
capabilities for planetary exploration for more than a decade.  First
demonstrated during the Mars Pathfinder Mission, the IRG has delivered
tools providing these capabilities to the science operations teams of
the Mars Polar Lander (MPL) mission, the Mars Exploration Rover (MER)
mission, and most recently the Mars Reconaissance Orbiter Mission. A
critical component technology enabling this work is the Ames Stereo
Pipeline (ASP).  ASP generates high quality, dense, texture-mapped
3D surface models from stereo image pairs.

Although initially developed for ground control and scientific
visualization applications, the ASP has evolved in recent years to
address orbital stereogrammetry and cartographic applications.  In
particular, long-range mission planning requires detailed knowledge of
planetary topography, and high resolution topography is often derived
from stereo pairs captured from orbit.  Orbital mapping satellites are
sent as precursors to planetary bodies in advance of landers and
rovers.  They return a wealth of imagery and other data that helps
mission planners and scientist identify areas worthy of more detailed
study. Topographic information often plays a central role in this
planning and analysis process.

Laser range-finding instruments such as the Mars Orbital Laser
Altimeter (MOLA) \citep{1992JGR....97.7781Z,2001JGR...10623689S}
has significantly advanced the study of the Martian surface by
providing geologists with a highly accurate elevation map of the
entire planet.  The upcoming Lunar Orbital Laser Altimeter (LOLA)
\citep{2008AGUFM.P31B1419N,2007SSRv..129..391C} will achieve similar
results for the Moon.  However, MOLA topographic products are limited
in resolution (463-m/pixel at the equator).  This, coupled with
localized interpolation artifacts in some regions due to sparse
laser data, have rendered MOLA products insufficient for detailed
studies of specific sites; e.g. geologic stratification and deposition
analysis, or in the case of mission planning, landing site selection.

The most common technique for obtaining higher-resolution digital
terrain models (DTMs) is to employ stereogrammetric techniques.
However, existing processing techniques on photogrammetric
workstations are extremely human intensive and expensive.  The
substantial number of man-hours and resources required for this
aproach has meant that relatively few of the existing stereo image
pairs have been processed, hence very few of these data products have
reached the scientific community.

The ASP was designed to address this shortfall by automating the
stereo reconstruction process so that it can run without human
guidance.  By applying recent advances in robotics and computer
vision, we have created an automated process that is capable of
generating high quality DEMs with minimal human intervention.  Users
of the Stereo Pipeline can expect to spend some time picking a handful
of settings when they first start processing a new type of imagery,
but once this is done the Stereo Pipeline can be used to process tens,
hundreds, or even thousands of stereo pairs without further
adjustment.

\section{Warnings to users of the Ames Stereo Pipeline ALPHA}

This is an ALPHA release of the Stereo Pipeline.  There are many known
bugs and incomplete features, and the API and command line options
will almost certainly change prior to the final release.  Much of the
documentation is incomplete or and some of it may be out of date or
incorrect.  Although we hope you will find this release helpful, you
use it at your own risk.

While we are confident that the algorithms used by this software are
robust, they have not been systematically tested or rigorously
compared to other methods in the peer-reviewed literature.  As such,
there is no guarantee that the topography produced by this software is
any good.  We have a number of efforts underway to carefully compare
Stereo Pipeline-generated data products to those produced by
established processes, and we will publish those results as they
become available.  In the meantime, we {\em strongly} recommend that
you consult us before publishing any results based on the cartographic
products produced by this software. You have been warned!



